% vim: foldmethod=marker foldmarker=(fold),(end)
\documentclass{article}

\usepackage{amsmath}
\usepackage{amsthm}
\usepackage{amssymb}
\usepackage{graphicx}
\usepackage{geometry}[margin=20mm]
\usepackage{listings}
\usepackage{svg}

\usepackage[numbers]{natbib}
\usepackage[colorlinks=true,linkcolor=blue,citecolor=blue,urlcolor=blue]{hyperref}

\def\c#1{\texttt{#1}}

\title{Quantum Information Analysis of Majorana Fermion Braiding: Exploring Realization and Error Assessment via a Series of Measurements}
\author{Alfaifi, Ammar -- 201855360}
\date{\today}

\begin{document}
\maketitle

\section{Instroduction} % (fold)
\label{sec:Instroduction}
In 1937, Majorana proposed a groundbreaking idea suggesting that electrically neutral spin-1/2 particles could be accurately described by a wave equation with real-valued solutions, known as the Majorana equation \cite{Majorana2006}. This proposition introduced the concept that these particles and their antiparticles would be indistinguishable due to the unique relationship between their wave functions through complex conjugation, ultimately leaving the Majorana wave equation unaltered.

Building upon Majorana's pioneering work, this paper delves into the fascinating realm of Majorana fermion braiding, transitioning from conventional approaches to a novel method centered around a sequence of measurements. A key aspect of this exploration involves leveraging the Jordan-Wigner transformation to express Majorana operators within the framework of a spin (fermionic) system. This transformative step not only enhances the understanding of Majorana fermions but also opens up possibilities for utilizing advanced techniques provided by the \c{qiskit} package.

Expanding the discourse, it is imperative to delve into the realm of quantum commuting and its inherent power. Quantum commuting, a phenomenon where quantum particles exchange positions without any physical trace, plays a pivotal role in the dynamics of quantum systems. This capability is harnessed in various quantum computing paradigms, and understanding its nuances is crucial for the successful realization of Majorana fermion braiding.

Furthermore, the introduction of the braiding operator adds another layer of complexity and significance to the discussion. The braiding operator is a mathematical entity that encapsulates the non-Abelian statistics associated with the braiding of anyons, a fundamental concept in the study of topological quantum computation. Exploring the properties and applications of the braiding operator provides valuable insights into the behavior of Majorana fermions during braiding processes.

Fault-tolerant quantum computing is of particular interest due to the inherent challenges posed by the delicate nature of quantum information. Unlike classical computation, which can often rely on the reliability of classical gates, quantum systems are highly susceptible to errors and decoherence. In the classical realm, information storage on magnetic media, for instance, benefits from the collective behavior of spins in individual atoms. Although each spin is sensitive to thermal fluctuations, the interactions between spins tend to align them in the same direction. Consequently, if a spin deviates, the interactions compel it to return to the alignment of other spins, resembling a self-correcting mechanism akin to the repetition code in error correction.\cite{Kitaev_2003}

Motivation behind utilizing a measurement-based protocol for realizing the braiding operator deserves attention. By adopting a measurement-based approach, the study aims to exploit the inherent advantages offered by this methodology, such as fault tolerance and scalability. Understanding the motivation behind choosing this protocol sheds light on the strategic decisions made in the pursuit of implementing Majorana fermion braiding.

Topological properties play a pivotal role in hosting quantum computation and Majorana braiding. The unique characteristics of topological systems, such as robustness against local perturbations, make them ideal candidates for quantum information processing. Examining why these properties are interesting in the context of Majorana braiding provides a comprehensive understanding of the broader implications and applications in quantum computing.

In the context of quantum gates simulation, the discussion extends to the utilization of measurement-only approaches for simulating quantum gates. Specifically, exploring how measurement-only strategies can effectively simulate quantum gates is crucial for comprehending the intricacies involved in the proposed Majorana fermion braiding scheme. Drawing parallels to existing quantum computing platforms, like the IBM Q computer, underscores the practicality and feasibility of implementing such schemes in real-world quantum hardware.

The exploration of accessing and running circuits on platforms such as the IBM Q computer is a crucial aspect. It is imperative to comprehend the processes and mechanisms involved in accessing and executing circuits on quantum computing hardware, as this knowledge forms the foundation for translating theoretical advancements into practical applications. In this report, we successfully employed the \c{qiskit} library and IBM online quantum computers to simulate our conceptual framework. This involved the creation of a two-qubit system that realizes our sequence of measurements through Pauli unitary gates. Through a comparison with a true braiding operator for a single qubit system, we assessed the results and error sensitivity of such circuits by calculating the average fidelity between the final state vector of the circuit after execution.
% section Instroduction (end)

\section{Majorana Definition} % (fold)
\label{sec:Majorana Definition}
The distinction between Majorana fermions and Dirac fermions can be mathematically articulated through the creation and annihilation operators of second quantization. Specifically, the creation operator $\gamma_j^{\dagger}$ generates a fermion in quantum state $j$ (described by a real wave function), while the annihilation operator $\gamma_j$ eliminates it, or equivalently, creates the corresponding antiparticle. In the case of a Dirac fermion, the operators $\gamma_j^{\dagger}$ and $\gamma_j$ are distinct, whereas for a Majorana fermion, they are identical. Expressing the ordinary fermionic annihilation and creation operators $f$ and $f^{\dagger}$ in terms of two Majorana operators $\gamma_1$ and $\gamma_2$ can be achieved as follows:

$$
	\begin{aligned}
		 & f=\frac{1}{\sqrt{2}}\left(\gamma_1+i \gamma_2\right),            \\
		 & f^{\dagger}=\frac{1}{\sqrt{2}}\left(\gamma_1-i \gamma_2\right) .
	\end{aligned}
$$

Having an even number of Majorana fermions, $ 2n $, they obey the anitcommutation relation
\begin{equation}
	\left\{\gamma_i, \gamma_j\right\}=2 \delta_{i j}, \qquad i=1, 2, \dots, 2n  .
\end{equation}
Mathematically, the superconductor imposes electron hole "symmetry" on the quasiparticle excitations, relating the creation operator $\gamma(E)$ at energy $E$ to the annihilation operator $\gamma^{\dagger}(-E)$ at energy $-E$.\cite{Majorana-returns} Majorana fermions can be bound to a defect at zero energy, and then the combined objects are called Majorana bound states or Majorana zero modes. This name is more appropriate than Majorana fermion (although the distinction is not always made in the literature), because the statistics of these objects is no longer fermionic. Instead, the Majorana bound states are an example of non-abelian anyons: interchanging them changes the state of the system in a way that depends only on the order in which the exchange was performed. The non-abelian statistics that Majorana bound states possess allows them to be used as a building block for a topological quantum computer.
% section Majorana Definition (end)

\section{Non-abelian Theory} % (fold)
\label{sec:Non-abelian Theory}

Consider the wavefunctions $ \psi_1 $ and $ \psi_2 $ associated with two particles. Upon applying an exchange operator, one anticipates the form
\begin{equation}
	\hat{E} | \psi_1\, \psi_2 \rangle = \lambda | \psi_2\, \psi_1 \rangle,
\end{equation}
where for fermions, $ \lambda = -1 $, and for bosons, $ \lambda = +1 $—both being constant. However, in the case of Majorana fermions, as we will elaborate on shortly, the constant takes the form of a matrix. This distinction categorizes the former as abelian particles and the latter as non-abelian particles \cite{Nayak_2008}.

- parity operator has +1 and -1 eigenvalues, hence its name. 
- the state and space of its eigenstate can be trated as if it's qubit to be used as, like utilizing the Hilbert space as if it a subspace of a quit, then using it to store and do quantum computation.
- it's non-local operator, since its property can be very away seperatred, giving it robustness against error and so on.
- fact no of particle is conserved, unitary operators that we can apply are limited, to those leave total parity conserved, so we found out the braiding abid to this conservation.

Any Hermitian operator $ A $ can be expressed as a unitary operator of the form $ U = e^{i\beta A} $ with some angle $ \beta $. For Majorana fermion operators, we can begin with their parity operator, denoted as $ P_{nm} = i \gamma_n \gamma_m $. 
the parity operator plays a crucial role in describing the non-abelian statistics associated with the braiding of these particles. The braiding of Majorana fermions involves exchanging their positions in space, and the resulting transformation is characterized by non-trivial matrix elements rather than simple constants.

This allows us to define
\begin{equation}
	U \equiv e^{\beta \gamma_n \gamma_m} \quad \text{or} \quad U = \cos{\beta} + \gamma_n \gamma_m \sin{\beta}
	\label{eq:U in beta}
\end{equation}

To find a unitary operator that evolves Majorana fermions according to the Heisenberg picture, i.e., 
$$
	\begin{aligned}
		\gamma_n \rightarrow U \gamma_n U ^\dagger, \\
		\gamma_m \rightarrow U \gamma_m U ^\dagger,
	\end{aligned}
$$
while leaving other elements unaffected, we substitute Equation~\ref{eq:U in beta} into the transformation equation:
$$
	\begin{aligned}
		\gamma_n \rightarrow \cos{2 \beta} \gamma_n - \sin{2 \beta} \gamma_m, \\
		\gamma_m \rightarrow \cos{2 \beta} \gamma_m - \sin{2 \beta} \gamma_n.
		\label{eq:subst in transformation}
	\end{aligned}
$$

This leads to the condition $ \beta = \pm \pi / 4 $. Substituting this into the expression, we obtain the braiding unitary operator for $ \gamma_n $ and $ \gamma_m $:
$$
	\begin{aligned}
		U = \exp \left( \pm \frac{\pi}{4} \gamma_n \gamma_m\right) = \frac{1}{\sqrt{2}}\left(1 \pm \gamma_n \gamma_m\right).
	\end{aligned}
$$
% section Non-abelian Theroy (end)

\section{Realizations of Braiding Via a Series of Measurements} % (fold)
\label{sec:Realizations of Braiding Via a Series of Measurements}

Consider the configuration of Majorana fermions depicted in Figure~\ref{fig:4 mf conf}. And, say, we want to braid the two Majorana fermions $ \gamma_0 $ \& $ \gamma_3 $. The conceptual approach here involves executing measurement-only operations on the configuration of Majorana modes, effectively achieving the same outcome as physically braiding the two Majorana fermions \cite{Leijnse_2012}.

\begin{figure}
	\begin{center}
		\includesvg[width=0.2\linewidth]{./figures/conf.svg}
	\end{center}
	\caption{Configuration of four Majorana fermions.}
	\label{fig:4 mf conf}
\end{figure}

To illustrate this idea, we begin with a system of 4 Majorana fermions, corresponding to two fermions. The configuration of Majorana fermions is depicted in Figure~\ref{fig:4 mf conf}. The true braiding operator between $\gamma_0$ and $\gamma_3$ is given by
\begin{equation}
	U = e^{\frac{\pi}{4} \gamma_0 \gamma_3}
	\label{eq:braiding op}
\end{equation}

To realize this braiding operator solely through a series of measurements, we follow these four steps:

\begin{enumerate}
	\item Apply the operator $ (1 + i \gamma_1 \gamma_2) $
	\item Apply the operator $ (1 + i \gamma_1 \gamma_0) $
	\item Apply the operator $ (1 + i \gamma_3 \gamma_1) $
	\item Apply the operator $ (1 + i \gamma_1 \gamma_2) $
\end{enumerate}
Which all are illustrated in Figure~\ref{fig:meas 4 mfs}.
\begin{figure}
	\begin{center}
		\includesvg[width=0.7\linewidth]{./figures/conf-meas.svg}
	\end{center}
	\caption{a-b shows measurements sequence that evantually realizes braiding of $ \gamma_0 $ and $ \gamma_3 $.}
	\label{fig:meas 4 mfs}
\end{figure}
% section Realizations of Braiding Via a Series of Measurements (end)

\section{Jordan-Wigner Transformation} % (fold)
\label{sec:Jordan-Wigner Transformation}
We shall redefine the our $ \gamma $ s in term of fermionic spin operators, giving us a way to model
this system in much more fimiliar systems, such as qubits in quantum computing information. So we'll have:
\begin{itemize}
	\item $ \gamma_0 = Z_0 $
	\item $ \gamma_1 = X_0 Z_1 $
	\item $ \gamma_2 = X_0 Y_1 $
	\item $ \gamma_3 = Y_0 $
\end{itemize}
Note: tensor product is understod, if there is one gate, tensor product with indentity of that subsystem is implicit.
Then 4-step series of measurement on the system becomes
\begin{enumerate}
	\item $ (1 + i X_0 Z_1 X_0 Y_1) = (1 + X_1) $
	\item $ (1 + i X_0 Z_1 Z_0) = (1 + Y_0 Z_1) $
	\item $ (1 + i Y_0 X_0 Z_1) = (1 + Z_0 Z_1) $
	\item $ (1 + X_1) $
\end{enumerate}
Also for true braiding operator we get
$$
	\begin{aligned}
		e^{i \frac{\pi}{4} X_0} = \frac{1}{\sqrt{2}}\, (1 + i X_0 ) \quad \text{or} \quad
		e^{-i \frac{\pi}{4} X_0} = \frac{1}{\sqrt{2}}\, (1 - i X_0 )
		\label{eq:br Jordan-Wigner}
	\end{aligned}
$$
% section Jordan-Wigner Transformation (end)

\section{Applying all Measurements} % (fold)
\label{sec:Applying all Measurements}
Let's understand the possible outcomes from the general case of the measurement operator,
that is,
$$
	\begin{aligned}
		(1 +S_3 X_1) (1 +S_2 Z_0 Z_1) (1 +S_1 Y_0 Z_1) (1 +S_0 X_1)
		\label{eq:general meas}
	\end{aligned}
$$
Expanding the middle two factors as
$$
	\begin{aligned}
		(1 +S_3 X_1) (1 + S_2 Z_0 Z_1 + S_1 Y_0 Z_1 + S_2 S_1 Z_0 Z_1 Y_0 Z_1) (1 +S_0 X_1)
		\label{eq:expanding}
	\end{aligned}
$$
Utilizing the Pauli gates anitcommutation relations, we move the LHS factor to RHS,
as for first term we get
\begin{equation*}
	(1 +S_3 X_1) (1 +S_0 X_1) = \delta_{S_0,S_3}\, (1 +S_0 X_1)
\end{equation*}
For second term,
\begin{equation*}
	(1 +S_3 X_1) S_2 Z_0 Z_1 (1 +S_0 X_1) = \delta_{S_0,-S_3}\, S_2 Z_0 Z_1 (1 +S_0 X_1)
\end{equation*}
For the 3rd term,
\begin{equation*}
	(1 +S_3 X_1) S_1 Y_0 Z_1 (1 +S_0 X_1) = \delta_{S_0,-S_3}\, S_1 Y_0 Z_1 (1 +S_0 X_1)
\end{equation*}
For the 3rd term,
\begin{equation*}
	(1 +S_3 X_1) S_2 S_1 Z_0 Z_1 Y_0 Z_1 (1 +S_0 X_1) = \delta_{S_0,S_3}\, -i X_0 S_2 S_1 (1 +S_0 X_1)
\end{equation*}
% section Applying all Measurements (end)

\section{Constructing Protocol} % (fold)
\label{sec:Constructing Protocol}
Now, we'll investigate the protocol classic outcomes, then we shall decide based on it whether
we did relize a braiding between $ \gamma_0 $ \& $ \gamma_3 $, if not, what operators
to apply to fix it. From Section~\ref{sec:Applying all Measurements}, we simplify it to
\begin{equation*}
	[
		\delta_{S_0,S_3} + \delta_{S_0,-S_3}\, S_2 Z_0 Z_1
		+ \delta_{S_0,-S_3}\, S_1 Y_0 Z_1 + \delta_{S_0,S_3}\, -i X_0 S_2 S_1
	] (1 +S_0 X_1)
\end{equation*}

Let's study different cases:
\begin{description}
	\item[Case 1: $ S_0 = S_3 $]
	      We get \begin{equation*}
		      [1 -i X_0 S_2 S_1] (1 + S_0 X_1)
	      \end{equation*}
	      Note, the right factor just acts on subsystem 1 that we don't care about
	      it outcomes.

	      \begin{description}
		      \item[Case 1.1: $ S_1 = -S_2 $]
		            \begin{equation*}
			            [1 + i X_0] (1 + S_0 X_1)
		            \end{equation*}
		            relizing counterclockwise braiding operator in Equation~\ref{eq:br Jordan-Wigner}.
		      \item[Case 1.2: $ S_1 = S_2 $]
		            \begin{equation*}
			            [1 - i X_0] (1 + S_0 X_1)
		            \end{equation*}
		            relizing clockwise braiding operator in Equation~\ref{eq:br Jordan-Wigner}.
	      \end{description}

	\item[Case 2: $ S_0 \ne S_3 $]
	      We get \begin{equation*}
		      [S_2 Z_0 Z_1 + S_1 Y_0 Z_1]\, (1 + S_0 X_1)
	      \end{equation*}
	      let's factor out $ Z_0 Z_1 $
	      \begin{equation*}
		      Z_0 Z_1 [S_2 - i S_1 X_0]\, (1 + S_0 X_1)
	      \end{equation*}

	      In this case we always want to multiply by $ Z_0 $, then we'll have
	      \begin{description}
		      \item[Case 2.1: $ S_1 = S_2 $]
		            \begin{equation*}
			            S_1 Z_0 Z_1 [1 - i X_0] (1 + S_0 X_1)
		            \end{equation*}
		            relizing the inverse braiding operator
		      \item[Case 2.2: $ S_1 = - S_2 $]
		            \begin{equation*}
			            S_1 Z_0 Z_1 [S_1 S_2 - i X_0] (1 + S_0 X_1)
		            \end{equation*}
		            But $ S_1 S_2 = -1 $, then
		            \begin{equation*}
			            - S_1 Z_0 Z_1 [1 + i X_0] (1 + S_0 X_1)
		            \end{equation*}
		            relizing the braiding operator
	      \end{description}
\end{description}
% section Constructing Protocol (end)

\section{Statevector and Density Matrix} % (fold)
\label{sec:Statevector and Density Matrix}
Assume you have a state vector $|\psi\rangle$, which is a column vector with complex numbers:
\begin{align}
	|\psi\rangle = \begin{bmatrix} \psi_1 \\ \psi_2 \\ \vdots \\ \psi_n \end{bmatrix}
\end{align}
Here, $n$ is the dimensionality of the quantum system.
Take the conjugate transpose of the state vector, denoted by $\langle \psi|$. This is obtained by taking the complex conjugate of each element and then transposing the vector:
\begin{align}
	\langle \psi| = \begin{bmatrix} \psi_1^* & \psi_2^* & \ldots & \psi_n^* \end{bmatrix}^T
\end{align}
Compute the outer product to obtain the density matrix $\rho$:
\begin{align}
	\rho = |\psi\rangle \langle \psi| = \begin{bmatrix} \psi_1 \\ \psi_2 \\ \vdots \\ \psi_n \end{bmatrix} \begin{bmatrix} \psi_1^* & \psi_2^* & \ldots & \psi_n^* \end{bmatrix}
\end{align}
The result is a square matrix of size $n \times n$, representing the density matrix.
The density matrix $\rho$ is Hermitian (equal to its conjugate transpose) and positive semi-definite. If the state vector $|\psi\rangle$ is normalized (has a magnitude of 1), then the trace of the density matrix is equal to 1, which is a requirement for a valid density matrix representing a physical state.


\textbf{Pure State:}

For a pure state density matrix $\rho$, which satisfies $\rho^2 = \rho$ (idempotent), you can find the state vector $|\psi\rangle$ by:
\[ |\psi\rangle = \sqrt{\lambda} \cdot |\phi\rangle \]
Here, $\lambda$ is the non-zero eigenvalue of $\rho$, and $|\phi\rangle$ is the corresponding eigenvector.

\textbf{Mixed State:}
For a mixed state density matrix $\rho$ with multiple non-zero eigenvalues, it is not possible to uniquely determine a single state vector. The system is in a statistical mixture of pure states. However, you can find a set of state vectors and their corresponding probabilities:
\[ \rho = \sum_i p_i |\psi_i\rangle\langle\psi_i| \]
Here, $|\psi_i\rangle$ are the eigenvectors of $\rho$, and $p_i$ are the corresponding eigenvalues. The state vectors represent the pure states in the mixture, and the probabilities $p_i$ give the weight of each pure state in the mixture.

\textbf{Implementation:}
To implement this in practice, you can use a numerical linear algebra library (such as NumPy for Python) to compute the eigenvalues and eigenvectors of the density matrix. Depending on the specific form of the density matrix, you might need to use different methods.
Keep in mind that this process may not be unique, especially for mixed states, where different sets of pure states and probabilities can lead to the same density matrix.
% section Statevector and Density Matrix (end)

\newpage
\section{What's Next ?}

\newpage
\bibliographystyle{plain}
\bibliography{references}
\end{document}
