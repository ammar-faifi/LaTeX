% vim: foldmethod=marker foldmarker=(fold),(end)
\documentclass{article}

\usepackage{amsmath}
\usepackage{amsthm}
\usepackage{amssymb}
\usepackage{graphicx}
\usepackage{geometry}[margin=20mm]
\usepackage{listings}
\usepackage{svg}


\def\c#1{\texttt{#1}}

\title{Quantum Information Analysis of Majorana Fermion Braiding: Exploring Realization and Error Assessment in Series of Measurements}
\author{Alfaifi, Ammar -- 201855360}
\date{\today}

\begin{document}
\maketitle

\section{Instroduction} % (fold)
\label{sec:Instroduction}
The inception of the concept dates back to Majorana's proposal in 1937, wherein he postulated that electrically neutral spin-1/2 particles could be characterized by a real-valued wave equation known as the Majorana equation. According to this proposition, these particles would be indistinguishable from their antiparticles due to the relationship between their wave functions through complex conjugation, thereby leaving the Majorana wave equation unaltered.

This paper explores the realization of Majorana fermion braiding, shifting from the traditional approach to a series of measurements. Additionally, it employs the Jordan-Wigner transformation to express Majorana operators in the language of a spin (fermionic) system. By doing so, the study facilitates the utilization of techniques offered by the \c{qiskit} package to simulate and analyze this novel approach and its associated quantum system.
% section Instroduction (end)

\section{Majorana Definition} % (fold)
\label{sec:Majorana Definition}
The distinction between Majorana fermions and Dirac fermions can be mathematically articulated through the creation and annihilation operators of second quantization. Specifically, the creation operator $\gamma_j^{\dagger}$ generates a fermion in quantum state $j$ (described by a real wave function), while the annihilation operator $\gamma_j$ eliminates it, or equivalently, creates the corresponding antiparticle. In the case of a Dirac fermion, the operators $\gamma_j^{\dagger}$ and $\gamma_j$ are distinct, whereas for a Majorana fermion, they are identical. Expressing the ordinary fermionic annihilation and creation operators $f$ and $f^{\dagger}$ in terms of two Majorana operators $\gamma_1$ and $\gamma_2$ can be achieved as follows:

$$
	\begin{aligned}
		 & f=\frac{1}{\sqrt{2}}\left(\gamma_1+i \gamma_2\right),            \\
		 & f^{\dagger}=\frac{1}{\sqrt{2}}\left(\gamma_1-i \gamma_2\right) .
	\end{aligned}
$$

Having an even number of Majorana fermions, $ 2n $, they obey the anitcommutation relation
\begin{align}
	\left\{\gamma_i, \gamma_j\right\}=2 \delta_{i j}, \qquad i=1, 2, \dots, 2n  .
\end{align}
Mathematically, the superconductor imposes electron hole "symmetry" on the quasiparticle excitations, relating the creation operator $\gamma(E)$ at energy $E$ to the annihilation operator $\gamma^{\dagger}(-E)$ at energy $-E$. Majorana fermions can be bound to a defect at zero energy, and then the combined objects are called Majorana bound states or Majorana zero modes. This name is more appropriate than Majorana fermion (although the distinction is not always made in the literature), because the statistics of these objects is no longer fermionic. Instead, the Majorana bound states are an example of non-abelian anyons: interchanging them changes the state of the system in a way that depends only on the order in which the exchange was performed. The non-abelian statistics that Majorana bound states possess allows them to be used as a building block for a topological quantum computer.
% section Majorana Definition (end)

\section{Non-abelian Theroy} % (fold)
\label{sec:Non-abelian Theroy}
Any Hermitian operator $ A $ can be written as unitary operator in form $ U = e^{i\beta A} $ to some angle $ \beta $.
For a Majorana fermions operator, we can start from their parity operator, which is Hermitian, $ P_{nm} = i \gamma_n \gamma_m $
So we can have
\begin{align}
	U \equiv e^{\beta \gamma_n \gamma_m} \quad \text{or} \quad U = \cos{\beta} + \gamma_n \gamma_m \sin{\beta}
	\label{eq:U in beta}
\end{align}
We'll try to find a unitary operator that evolve Majorana fermion as following, in Heisenberg picture,
\begin{align}
	\gamma_n \rightarrow U \gamma_n U ^\dagger \\
	\gamma_m \rightarrow U \gamma_m U ^\dagger
	\label{eq:transform gammas}
\end{align}
Putting Equation~\ref{eq:U in beta} in Equation~\ref{eq:transform gammas} we get
\begin{align}
	\gamma_n \rightarrow \cos{2 \beta} \gamma_n - \sin{2 \beta} \gamma_m, \\
	\gamma_n \rightarrow \cos{2 \beta} \gamma_m - \sin{2 \beta} \gamma_n.
	\label{eq:subst in transformation}
\end{align}
Then we should have $ \beta = \pm \pi / 4 $. The we get braiding unitary operator of $ \gamma_n $ \& $ \gamma_m $
\begin{align}
	U = \exp \left( \pm \frac{\pi}{4} \gamma_n \gamma_m\right)
	=\frac{1}{\sqrt{2}}\left(1 \pm \gamma_n \gamma_m\right)
\end{align}
% section Non-abelian Theroy (end)

\section{Majorana Zero Modes} % (fold)
\label{sec:Majorana Zero Modes}
Assume we have the configuration of Majorana fermions shown in Figure~\ref{fig:4 mf conf}.
The idea now is to do measurement only operations on the Majorana modes' configuration that will realize
the same result as if we braid the two Majorana fermions.\cite{Leijnse_2012}
\begin{figure}
	\begin{center}
		\includesvg[width=0.3\linewidth]{./figures/conf.svg}
	\end{center}
	\caption{Four Majorana fermions configuration.}
	\label{fig:4 mf conf}
\end{figure}

% section Majorana Zero Modes (end)

\section{As Series of Measurement} % (fold)
\label{sec:As Series of measurement}
As a demostration for the idea, we'll start with a system of 4 Majorana fermions,
corresponding to two fermions. The configuration of Majorana fermions is shown in fig.
The true braiding operator between $\gamma_0$ and $ \gamma_3 $ is given by
\begin{align}
	U = e^{\frac{\pi}{4} \gamma_0 \gamma_3}
	\label{eq:braiding op}
\end{align}
Then, to relize this braiding operator as just series of measurement we do this in four steps:
\begin{enumerate}
	\item $ (1 + i \gamma_1 \gamma_2) $
	\item $ (1 + i \gamma_1 \gamma_0) $
	\item $ (1 + i \gamma_3 \gamma_1) $
	\item $ (1 + i \gamma_1 \gamma_2) $
\end{enumerate}
% section As Series of measurement (end)

\section{Jordan-Wigner Transformation} % (fold)
\label{sec:Jordan-Wigner Transformation}
We shall redefine the our $ \gamma $ s in term of fermionic spin operators, giving us a way to model
this system in much more fimiliar systems, such as qubits in quantum computing information. So we'll have:
\begin{itemize}
	\item $ \gamma_0 = Z_0 $
	\item $ \gamma_1 = X_0 Z_1 $
	\item $ \gamma_2 = X_0 Y_1 $
	\item $ \gamma_3 = Y_0 $
\end{itemize}
Note: tensor product is understod, if there is one gate, tensor product with indentity of that subsystem is implicit.
Then 4-step series of measurement on the system becomes
\begin{enumerate}
	\item $ (1 + i X_0 Z_1 X_0 Y_1) = (1 + X_1) $
	\item $ (1 + i X_0 Z_1 Z_0) = (1 + Y_0 Z_1) $
	\item $ (1 + i Y_0 X_0 Z_1) = (1 + Z_0 Z_1) $
	\item $ (1 + X_1) $
\end{enumerate}
Also for true braiding operator we get
\begin{align}
	e^{i \frac{\pi}{4} X_0} = \frac{1}{\sqrt{2}}\, (1 + i X_0 ) \quad \text{or} \quad
	e^{-i \frac{\pi}{4} X_0} = \frac{1}{\sqrt{2}}\, (1 - i X_0 )
	\label{eq:br Jordan-Wigner}
\end{align}
% section Jordan-Wigner Transformation (end)

\section{Applying all Measurements} % (fold)
\label{sec:Applying all Measurements}
Let's understand the possible outcomes from the general case of the measurement operator,
that is,
\begin{align}
	(1 +S_3 X_1) (1 +S_2 Z_0 Z_1) (1 +S_1 Y_0 Z_1) (1 +S_0 X_1)
	\label{eq:general meas}
\end{align}
Expanding the middle two factors as
\begin{align}
	(1 +S_3 X_1) (1 + S_2 Z_0 Z_1 + S_1 Y_0 Z_1 + S_2 S_1 Z_0 Z_1 Y_0 Z_1) (1 +S_0 X_1)
	\label{eq:expanding}
\end{align}
Utilizing the Pauli gates anitcommutation relations, we move the LHS factor to RHS,
as for first term we get
\begin{equation*}
	(1 +S_3 X_1) (1 +S_0 X_1) = \delta_{S_0,S_3}\, (1 +S_0 X_1)
\end{equation*}
For second term,
\begin{equation*}
	(1 +S_3 X_1) S_2 Z_0 Z_1 (1 +S_0 X_1) = \delta_{S_0,-S_3}\, S_2 Z_0 Z_1 (1 +S_0 X_1)
\end{equation*}
For the 3rd term,
\begin{equation*}
	(1 +S_3 X_1) S_1 Y_0 Z_1 (1 +S_0 X_1) = \delta_{S_0,-S_3}\, S_1 Y_0 Z_1 (1 +S_0 X_1)
\end{equation*}
For the 3rd term,
\begin{equation*}
	(1 +S_3 X_1) S_2 S_1 Z_0 Z_1 Y_0 Z_1 (1 +S_0 X_1) = \delta_{S_0,S_3}\, -i X_0 S_2 S_1 (1 +S_0 X_1)
\end{equation*}
% section Applying all Measurements (end)

\section{Constructing Protocol} % (fold)
\label{sec:Constructing Protocol}
Now, we'll investigate the protocol classic outcomes, then we shall decide based on it whether
we did relize a braiding between $ \gamma_0 $ \& $ \gamma_3 $, if not, what operators
to apply to fix it. From Section~\ref{sec:Applying all Measurements}, we simplify it to
\begin{equation*}
	[
		\delta_{S_0,S_3} + \delta_{S_0,-S_3}\, S_2 Z_0 Z_1
		+ \delta_{S_0,-S_3}\, S_1 Y_0 Z_1 + \delta_{S_0,S_3}\, -i X_0 S_2 S_1
	] (1 +S_0 X_1)
\end{equation*}

Let's study different cases:
\begin{description}
	\item[Case 1: $ S_0 = S_3 $]
	      We get \begin{equation*}
		      [1 -i X_0 S_2 S_1] (1 + S_0 X_1)
	      \end{equation*}
	      Note, the right factor just acts on subsystem 1 that we don't care about
	      it outcomes.

	      \begin{description}
		      \item[Case 1.1: $ S_1 = -S_2 $]
		            \begin{equation*}
			            [1 + i X_0] (1 + S_0 X_1)
		            \end{equation*}
		            relizing counterclockwise braiding operator in Equation~\ref{eq:br Jordan-Wigner}.
		      \item[Case 1.2: $ S_1 = S_2 $]
		            \begin{equation*}
			            [1 - i X_0] (1 + S_0 X_1)
		            \end{equation*}
		            relizing clockwise braiding operator in Equation~\ref{eq:br Jordan-Wigner}.
	      \end{description}

	\item[Case 2: $ S_0 \ne S_3 $]
	      We get \begin{equation*}
		      [S_2 Z_0 Z_1 + S_1 Y_0 Z_1]\, (1 + S_0 X_1)
	      \end{equation*}
	      let's factor out $ Z_0 Z_1 $
	      \begin{equation*}
		      Z_0 Z_1 [S_2 - i S_1 X_0]\, (1 + S_0 X_1)
	      \end{equation*}

	      In this case we always want to multiply by $ Z_0 $, then we'll have
	      \begin{description}
		      \item[Case 2.1: $ S_1 = S_2 $]
		            \begin{equation*}
			            S_1 Z_0 Z_1 [1 - i X_0] (1 + S_0 X_1)
		            \end{equation*}
		            relizing the inverse braiding operator
		      \item[Case 2.2: $ S_1 = - S_2 $]
		            \begin{equation*}
			            S_1 Z_0 Z_1 [S_1 S_2 - i X_0] (1 + S_0 X_1)
		            \end{equation*}
		            But $ S_1 S_2 = -1 $, then
		            \begin{equation*}
			            - S_1 Z_0 Z_1 [1 + i X_0] (1 + S_0 X_1)
		            \end{equation*}
		            relizing the braiding operator
	      \end{description}
\end{description}
% section Constructing Protocol (end)

\section{Statevector and Density Matrix} % (fold)
\label{sec:Statevector and Density Matrix}
Assume you have a state vector $|\psi\rangle$, which is a column vector with complex numbers:
\begin{align}
	|\psi\rangle = \begin{bmatrix} \psi_1 \\ \psi_2 \\ \vdots \\ \psi_n \end{bmatrix}
\end{align}
Here, $n$ is the dimensionality of the quantum system.
Take the conjugate transpose of the state vector, denoted by $\langle \psi|$. This is obtained by taking the complex conjugate of each element and then transposing the vector:
\begin{align}
	\langle \psi| = \begin{bmatrix} \psi_1^* & \psi_2^* & \ldots & \psi_n^* \end{bmatrix}^T
\end{align}
Compute the outer product to obtain the density matrix $\rho$:
\begin{align}
	\rho = |\psi\rangle \langle \psi| = \begin{bmatrix} \psi_1 \\ \psi_2 \\ \vdots \\ \psi_n \end{bmatrix} \begin{bmatrix} \psi_1^* & \psi_2^* & \ldots & \psi_n^* \end{bmatrix}
\end{align}
The result is a square matrix of size $n \times n$, representing the density matrix.
The density matrix $\rho$ is Hermitian (equal to its conjugate transpose) and positive semi-definite. If the state vector $|\psi\rangle$ is normalized (has a magnitude of 1), then the trace of the density matrix is equal to 1, which is a requirement for a valid density matrix representing a physical state.


\textbf{Pure State:}

For a pure state density matrix $\rho$, which satisfies $\rho^2 = \rho$ (idempotent), you can find the state vector $|\psi\rangle$ by:
\[ |\psi\rangle = \sqrt{\lambda} \cdot |\phi\rangle \]
Here, $\lambda$ is the non-zero eigenvalue of $\rho$, and $|\phi\rangle$ is the corresponding eigenvector.

\textbf{Mixed State:}
For a mixed state density matrix $\rho$ with multiple non-zero eigenvalues, it is not possible to uniquely determine a single state vector. The system is in a statistical mixture of pure states. However, you can find a set of state vectors and their corresponding probabilities:
\[ \rho = \sum_i p_i |\psi_i\rangle\langle\psi_i| \]
Here, $|\psi_i\rangle$ are the eigenvectors of $\rho$, and $p_i$ are the corresponding eigenvalues. The state vectors represent the pure states in the mixture, and the probabilities $p_i$ give the weight of each pure state in the mixture.

\textbf{Implementation:}
To implement this in practice, you can use a numerical linear algebra library (such as NumPy for Python) to compute the eigenvalues and eigenvectors of the density matrix. Depending on the specific form of the density matrix, you might need to use different methods.
Keep in mind that this process may not be unique, especially for mixed states, where different sets of pure states and probabilities can lead to the same density matrix.
% section Statevector and Density Matrix (end)

\bibliographystyle{plain}
\bibliography{references}
\end{document}
