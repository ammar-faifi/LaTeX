% vim: foldmethod=marker foldmarker=(fold),(end)
\documentclass{article}

\usepackage{amsmath}
\usepackage{amsthm}
\usepackage{amssymb}
\usepackage{graphicx}
\usepackage{geometry}[margin=20mm]

\def\c#1{\texttt{#1}}

\title{Physics Undergraduate Research (PHYS497)}
\author{Alfaifi, Ammar -- 201855360}
\date{Nov. 2023}

\begin{document}
\maketitle

\section{Instroduction} % (fold)
\label{sec:Instroduction}
This paper shows the idea of relizaion of braiding of Majorana fermions but as
series of measurement istead. And then using Jordan-Wigner transformation to
write Majorana operators in terms of spin (fermionic) system. Hence, we can use
methods provided by \c{qiskit} package to simulate such an idea and system.
% section Instroduction (end)

\section{As Series of Measurement} % (fold)
\label{sec:As Series of measurement}
As a demostration for the idea, we'll start with a system of 4 Majorana fermions,
corresponding to two fermions. The configuration of Majorana fermions is shown in fig.
The true braiding operator between $\gamma_0$ and $ \gamma_3 $ is given by
\begin{equation}
	U = e^{\frac{\pi}{4} \gamma_0 \gamma_3}
	\label{eq:braiding op}
\end{equation}
Then, to relize this braiding operator as just series of measurement we do this in four steps:
\begin{enumerate}
	\item $ (1 + i \gamma_1 \gamma_2) $
	\item $ (1 + i \gamma_1 \gamma_0) $
	\item $ (1 + i \gamma_3 \gamma_1) $
	\item $ (1 + i \gamma_1 \gamma_2) $
\end{enumerate}
% section As Series of measurement (end)

\section{Jordan-Wigner Transformation} % (fold)
\label{sec:Jordan-Wigner Transformation}
We shall redefine the our $ \gamma $ s in term of fermionic spin operators, giving us a way to model
this system in much more fimiliar systems, such as qubits in quantum computing information. So we'll have:
\begin{itemize}
	\item $ \gamma_0 = Z_0 $
	\item $ \gamma_1 = X_0 Z_1 $
	\item $ \gamma_2 = X_0 Y_1 $
	\item $ \gamma_3 = Y_0 $
\end{itemize}
Note: tensor product is understod, if there is one gate, tensor product with indentity of that subsystem is implicit.
Then 4-step series of measurement on the system becomes
\begin{enumerate}
	\item $ (1 + i X_0 Z_1 X_0 Y_1) = (1 + X_1) $
	\item $ (1 + i X_0 Z_1 Z_0) = (1 + Y_0 Z_1) $
	\item $ (1 + i Y_0 X_0 Z_1) = (1 + Z_0 Z_1) $
	\item $ (1 + X_1) $
\end{enumerate}
Also for true braiding operator we get
\begin{equation}
	e^{i \frac{\pi}{4} X_0} = \frac{1}{\sqrt{2}}\, (1 + i X_0 ) \quad \text{or} \quad
	e^{-i \frac{\pi}{4} X_0} = \frac{1}{\sqrt{2}}\, (1 - i X_0 )
	\label{eq:br Jordan-Wigner}
\end{equation}
% section Jordan-Wigner Transformation (end)

\section{Applying all Measurements} % (fold)
\label{sec:Applying all Measurements}
Let's understand the possible outcomes from the general case of the measurement operator,
that is,
\begin{equation}
	(1 +S_3 X_1) (1 +S_2 Z_0 Z_1) (1 +S_1 Y_0 Z_1) (1 +S_0 X_1)
	\label{eq:general meas}
\end{equation}
Expanding the middle two factors as
\begin{equation}
	(1 +S_3 X_1) (1 + S_2 Z_0 Z_1 + S_1 Y_0 Z_1 + S_2 S_1 Z_0 Z_1 Y_0 Z_1) (1 +S_0 X_1)
	\label{eq:expanding}
\end{equation}
Utilizing the Pauli gates anitcommutation relations, we move the LHS factor to RHS,
as for first term we get
\begin{equation*}
	(1 +S_3 X_1) (1 +S_0 X_1) = \delta_{S_0,S_3}\, (1 +S_0 X_1)
\end{equation*}
For second term,
\begin{equation*}
	(1 +S_3 X_1) S_2 Z_0 Z_1 (1 +S_0 X_1) = \delta_{S_0,-S_3}\, S_2 Z_0 Z_1 (1 +S_0 X_1)
\end{equation*}
For the 3rd term,
\begin{equation*}
	(1 +S_3 X_1) S_1 Y_0 Z_1 (1 +S_0 X_1) = \delta_{S_0,-S_3}\, S_1 Y_0 Z_1 (1 +S_0 X_1)
\end{equation*}
For the 3rd term,
\begin{equation*}
	(1 +S_3 X_1) S_2 S_1 Z_0 Z_1 Y_0 Z_1 (1 +S_0 X_1) = \delta_{S_0,S_3}\, -i X_0 S_2 S_1 (1 +S_0 X_1)
\end{equation*}
% section Applying all Measurements (end)

\section{Constructing Protocol} % (fold)
\label{sec:Constructing Protocol}
Now, we'll investigate the protocol classic outcomes, then we shall decide based on it whether
we did relize a braiding between $ \gamma_0 $ \& $ \gamma_3 $, if not, what operators
to apply to fix it. From Section~\ref{sec:Applying all Measurements}, we simplify it to
\begin{equation*}
	[
		\delta_{S_0,S_3} + \delta_{S_0,-S_3}\, S_2 Z_0 Z_1
		+ \delta_{S_0,-S_3}\, S_1 Y_0 Z_1 + \delta_{S_0,S_3}\, -i X_0 S_2 S_1
	] (1 +S_0 X_1)
\end{equation*}

Let's study different cases:
\begin{description}
	\item[Case 1: $ S_0 = S_3 $]
		We get \begin{equation*}
			[1 -i X_0 S_2 S_1] (1 + S_0 X_1)
		\end{equation*}
		Note, the right factor just acts on subsystem 1 that we don't care about
		it outcomes.

		\begin{description}
			\item[Case 1.1: $ S_1 = -S_2 $]
				\begin{equation*}
					[1 + i X_0] (1 + S_0 X_1)
				\end{equation*}
				relizing counterclockwise braiding operator in Equation~\ref{eq:br Jordan-Wigner}.
			\item[Case 1.2: $ S_1 = S_2 $]
				\begin{equation*}
					[1 - i X_0] (1 + S_0 X_1)
				\end{equation*}
				relizing clockwise braiding operator in Equation~\ref{eq:br Jordan-Wigner}.
		\end{description}

	\item[Case 2: $ S_0 \ne S_3 $]
		We get \begin{equation*}
			[S_2 Z_0 Z_1 + S_1 Y_0 Z_1]\, (1 + S_0 X_1)
		\end{equation*}
		let's factor out $ Z_0 Z_1 $
		\begin{equation*}
			Z_0 Z_1 [S_2 - i S_1 X_0]\, (1 + S_0 X_1)
		\end{equation*}

		In this case we always want to multiply by $ Z_0 $, then we'll have
		\begin{description}
			\item[Case 2.1: $ S_1 = S_2 $]
				\begin{equation*}
					S_1 Z_0 Z_1 [1 - i X_0] (1 + S_0 X_1)
				\end{equation*}
				relizing the inverse braiding operator
			\item[Case 2.2: $ S_1 = - S_2 $]
				\begin{equation*}
					S_1 Z_0 Z_1 [S_1 S_2 - i X_0] (1 + S_0 X_1)
				\end{equation*}
          But $ S_1 S_2 = -1 $, then 
          \begin{equation*}
            - S_1 Z_0 Z_1 [1 + i X_0] (1 + S_0 X_1)
          \end{equation*}
				relizing the braiding operator
		\end{description}
\end{description}
% section Constructing Protocol (end)

\end{document}
