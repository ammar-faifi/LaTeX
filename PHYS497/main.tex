% vim: foldmethod=marker foldmarker=(fold),(end)
\documentclass{article}

\usepackage{amsmath}
\usepackage{amsthm}
\usepackage{amssymb}
\usepackage{graphicx}

\def\c#1{\texttt{#1}}

\title{Physics Undergraduate Research (PHYS497)}
\author{Alfaifi, Ammar -- 201855360}
\date{Nov. 2023}

\begin{document}
\maketitle

\section{Instroduction} % (fold)
\label{sec:Instroduction}
This paper shows the idea of relizaion of braiding of Majorana fermions but as 
series of measurement istead. And then using Jordan-Wigner transformation to 
write Majorana operators in terms of spin (fermionic) system. Hence, we can use 
methods provided by \c{qiskit} package to simulate such an idea and system.
% section Instroduction (end)

\section{As Series of Measurement} % (fold)
\label{sec:As Series of measurement}
As a demostration for the idea, we'll start with a system of 4 Majorana fermions, 
corresponding to two fermions. The configuration of Majorana fermions is shown in fig. 
The true braiding operator between $\gamma_0$ and $ \gamma_3 $ is given by
\begin{equation}
  U = e^{\frac{\pi}{4} \gamma_0 \gamma_3}
  \label{eq:braiding op}
\end{equation}
Then, to relize this braiding operator as just series of measurement we do this in four steps:
\begin{enumerate}
  \item $ (1 + i \gamma_1 \gamma_2) $
  \item $ (1 + i \gamma_0 \gamma_1) $
  \item $ (1 + i \gamma_3 \gamma_1) $
  \item $ (1 + i \gamma_1 \gamma_2) $
\end{enumerate}
% section As Series of measurement (end)

\section{Jordan-Wigner Transformation} % (fold)
\label{sec:Jordan-Wigner Transformation}
We shall redefine the our $ \gamma $ s in term of fermionic spin operators, giving us a way to model 
this system in much more fimiliar systems, such as qubits in quantum computing information. So we'll have:
\begin{itemize}
  \item $ \gamma_0 = Z_0 $
  \item $ \gamma_0 = X_0 Z_1 $
  \item $ \gamma_0 = X_0 Y_1 $
  \item $ \gamma_0 = Y_0 $
\end{itemize}
Note: tensor product is understod, if there is one gate, tensor product with indentity of that subsystem is implicit.
Then 4-step series of measurement on the system becomes 
\begin{enumerate}
  \item $ (1 + X_1) $
  \item $ (1 - Y_0 Z_1) $
  \item $ (1 + Z_0 Z_1) $
  \item $ (1 + X_1) $
\end{enumerate}
Also for true braiding operator we get
\begin{equation}
  U = e^{i \frac{\pi}{4} X_0} = \frac{(1 + i X_0 )}{\sqrt{2}} 
  \label{eq:br Jordan-Wigner}
\end{equation}
% section Jordan-Wigner Transformation (end)

\section{Applying all Measurements} % (fold)
\label{sec:Applying all Measurements}
Let's understand the possible outcomes from the general case of the measurement operator, 
that is,
\begin{equation}
  (1 +S_0 X_1) (1 +S_1 Z_0 Z_1) (1 +S_2 Y_0 Z_1) (1 +S_3 X_1)
  \label{eq:general meas}
\end{equation}
Expanding the middle two factors as
\begin{equation}
  (1 +S_0 X_1) (1 + S_1 Z_0 Z_1 + S_2 Y_0 Z_1 + S_1 S_2 Z_0 Z_1 Y_0 Z_1) (1 +S_3 X_1)
  \label{eq:expanding}
\end{equation}
Utilizing the Pauli gates anitcommutation relations, we move the LHS factor to RHS, 
as for first term we get 
\begin{equation}
  (1 +S_0 X_1) (1 +S_3 X_1) = \delta_{S_0,S_3} (1 +S_3 X_1)
  \label{eq:after mv factor 1}
\end{equation}
For second term,
\begin{equation}
  (1 +S_0 X_1) S_1 Z_0 Z_1 (1 +S_3 X_1) = \delta_{-S_0,S_3} S_1 Z_0 Z_1 (1 +S_3 X_1)
  \label{eq:after mv factor 2}
\end{equation}
% section Applying all Measurements (end)

\end{document}
