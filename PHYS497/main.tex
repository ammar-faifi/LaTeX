\documentclass{paper}

\usepackage{amsmath}
\usepackage{amsthm}
\usepackage{amssymb}
\usepackage{graphicx}

\def\c#1{\texttt{#1}}

\title{Physics Undergraduate Research (PHYS497)}
\author{Alfaifi, Ammar -- 201855360}
\date{Nov. 11, 2023}

\begin{document}
\maketitle

\section{Instroduction} % (fold)
\label{sec:Instroduction}
This paper shows the idea of relizaion of braiding of Majorana fermions but as 
series of measurement istead. And then using Jordan-Wigner transformation to 
write Majorana operators in terms of spin (fermionic) system. Hence, we can use 
methods provided by \c{qiskit} package to simulate such an idea and system.
% section Instroduction (end)

\section{As Series of Measurement} % (fold)
\label{sec:As Series of measurement}
As a demostration for the idea, we'll start with a system of 4 Majorana fermions, 
corresponding to two fermions. 
% section As Series of measurement (end)

\end{document}
