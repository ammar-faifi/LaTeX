\documentclass{article}

\usepackage[a4paper, margin=15mm]{geometry}
\usepackage{amsmath}
\usepackage{amsthm}
\usepackage{amssymb}
\usepackage{graphicx}
\usepackage{enumitem}



\title{Homework 1 - Principles of Software Engineering (ICS431)}
\author{Alfaifi, Ammar}
\date{}

\begin{document}

\maketitle

\section*{Question 1}%

\begin{enumerate}[label=\alph*.]
  \item In waterfall model, the tasks are divides in smaller parts, and to be executed in
        five stages. This model is very suitable for projects the do not require to many versions suppers
        at once as well as very easy to understand or very complex details. Further this models lacks
        developing and providing the showcase uncompleted versions very frequently, so that
        it takes too long to have very first initial version. In other hand, it is a good
        model to provide high quality models; this is due to its slow developing process that helps
        catching errors in very early stages. Stakeholders involvement varies depending on the current
        stage of the project, for instance customers are only asked to use the product at the beginning of
        \textit{verification} stage. Finally, it eases the tracking of the progress since each stage is uniquely
        separated from its neighbor ones.

  \item We can think of \textit{incremental delivery} model as polarization of waterfall model. To
        explain, it allows the development of many version (\textit{increments}) of software at once
        which allows deliveries and/or maintaining multiple versions. It is suitable for projects that
        have UI and GUI to build, since developer need to keep validating its responsivity for many devices
        as well as for relatively small projects. That is, it provides very fast developing process.
        However, having many versions and many developers working simultaneously can lead to errors and
        bugs in the projects leading to lower quality outcomes, compared to waterfall model. Also, users,
        customers, and developers are all-time involved in the process. The feature of this models, it can
        show a very mature version of the software and collect feedback then re-run the process again and
        again.

  \item  The general idea of \textit{component-based software development} is dividing a systems into
        \underline{independent} subsystems. This workflow might be so hard to accomplish but it simplifies way
        much re-using any component in very different places in the software which speed up the developing
        process as well as keeping the overall good quality. Customers/stakeholders involvement is similar to
        the incremental model as well as for the visibility of each stage.

  \item In the \textit{rational unified process}, it has very similar concept with the waterfall
        model but the former provides a repeating process again and again. it is intended to be tailored by the
        development organizations and software project teams that will select the elements of the process
        that are appropriate for their needs. It is easy to apply due to its simple structure.


  \item Agile has is a way to model and document software systems based on best practices. This flexibility
        allows to choose whatever model to use such as Agile or RUP. This can increase the speed of
        delivering versions, with quite good quality outcomes. Stakeholders differs depending on the used
        model to rely on.

\end{enumerate}


\section*{Question 2}
\begin{enumerate}
  \item This requires many versions developed at the same time as well as having a quick initial version
        to test the UI and its responsive to many devices. Thus, incremental model is the best for this
        case.

  \item It has similar features to the one above, but this one requires high level of security and quality
        measures, not mention the entire process should all clear from the beginning. Then, they should
        follow the waterfall model.

  \item Those are independent subsystems that can be implemented as separated services then integrate them all
        as one entire system. Thus, component-based software development is the best here.

  \item It simple clear idea, with just one user interface for smartphones, so it better to use waterfall model.

\end{enumerate}


\end{document}
