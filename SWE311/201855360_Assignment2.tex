\documentclass{article}

\usepackage[a4paper, margin=15mm]{geometry}
\usepackage{amsmath}
\usepackage{amsthm}
\usepackage{amssymb}
\usepackage{graphicx}
\usepackage{enumitem}



\title{Homework 2 - Principles of Software Engineering (SWE311)}
\author{Alfaifi, Ammar}
\date{}

\begin{document}

\maketitle

\section*{Question 1}%

Free open-source components are often free but may lack quality and support,
while closed-source components are often reliable but can be expensive and
limit customization. The choice depends on the organization's needs and priorities.

\subsection*{Benefits of utilizing open-source components}
\begin{itemize}
	\item They are frequently free of charge, which can reduce the overall expenses of software development.
	\item The availability of the source code allows developers to adjust and personalize the component according to their requirements.
	\item A community of developers frequently contributes to the development and enhancement of the component.
\end{itemize}

\subsection*{Drawbacks of utilizing open-source components:}

\begin{itemize}
	\item The quality of the component can differ significantly since there is no assurance that it has undergone thorough testing or is dependable.
	\item The developer who created the component may not provide support or might be unavailable to address inquiries, resulting in limited support.
	\item Since open-source components are typically created independently, there may be compatibility issues with other components.
\end{itemize}

\subsection*{Advantages of utilizing closed-source components:}
\begin{itemize}

	\item Due to their financial interest, developers frequently test and guarantee the reliability of closed-source components.
	\item Developers frequently provide support, which can be useful when resolving problems.
	\item They are frequently designed to function seamlessly with other components from the same developer.
\end{itemize}

\subsection*{Drawbacks of utilizing closed-source components:}
\begin{itemize}

	\item They can be costly, which can raise the expenses of software development.
	\item The source code is not available, making it challenging to customize or modify the component to meet specific requirements.
	\item Using closed-source components from a specific developer may result in vendor lock-in, making it difficult to switch to another vendor or platform.
\end{itemize}

\section*{Question 2}%
\subsection*{Stakeholders}%

\begin{itemize}
	\item Patients
	\item Receptionist
	\item Doctors
	\item Nurses
	\item Clinic management
	\item Insurance companies
	\item Government health agencies
	\item Medical researchers and analysts
	\item IT support and maintenance team
\end{itemize}


\subsection*{Conflict requirements}
\begin{enumerate}
	\item Patients may ask for their personal details to be confidential and available only to authorized staff, but government regulators may require the clinic to disclose certain patient data for public health and safety reasons.
	\item Doctors may need fast and convenient access to patient records, while patients may demand that their medical information remain private and only accessible to authorized medical personnel. This could result in conflicting access control and authentication requirements to ensure that only authorized personnel can view patient records.
\end{enumerate}

\section{Question 3}%

\paragraph{Windows Vista's failure}
\begin{enumerate}



\item Compatibility issues: Many hardware and software products were not compatible with Windows Vista, which led to frustration among users.

\item Slow performance: Windows Vista required more resources than its predecessor, Windows XP, and ran slowly on many computers.

\item Lack of driver support: There was a lack of driver support for many hardware devices, which led to compatibility issues and poor performance.

\item Poor marketing: Microsoft did not do a good job of promoting the new features of Windows Vista, which led to confusion among users about its benefits.

\item High system requirements: Windows Vista required a relatively high-end computer to run smoothly, which made it inaccessible for many users.
\end{enumerate}

The failure of Windows Vista was \textbf{avoidable}. Microsoft could have taken several steps to prevent the issues that led to its failure. For example:


\begin{enumerate}

\item  Better testing: Microsoft could have done more rigorous testing before releasing the operating system to ensure that compatibility issues and performance problems were addressed.

\item  Improved marketing: Microsoft could have done a better job of promoting the new features of Windows Vista to help users understand the benefits of upgrading.

\item  Lower system requirements: Microsoft could have lowered the system requirements for Windows Vista, making it more accessible to a wider range of users.

\item More driver support: Microsoft could have worked with hardware manufacturers to ensure that there was adequate driver support for all devices.
\end{enumerate}

\end{document}
