\documentclass{article}

\usepackage{amsmath}
\usepackage{amsthm}
\usepackage{amssymb}
\usepackage{graphicx}
\usepackage{enumitem}
\usepackage{xcolor}
\usepackage{hyperref}
\usepackage{geometry}

\geometry{
  a4paper,
  left=25mm,
  right=25mm,
  top=25mm,
  bottom=25mm,
}

\hypersetup{
    colorlinks=true,
    linkcolor=blue,
    filecolor=magenta,
    urlcolor=blue,
}


\def\c#1{\texttt{#1}}

\title{Progress Report 2 - Summer Training (ICS399)}
\author{Alfaifi, Ammar - 201855360}
\date{\today}

\begin{document}

\maketitle

\section{Introduction}
During the last three weeks of my summer training in ICS399, I have been actively working as a software engineer, engaging in various tasks and gaining valuable experience in the field.

\section{Schedule}

\begin{tabular}{|c|c|c|c|}
	\hline
	\textbf{Task}                                       & \textbf{Start Date} & \textbf{Duration} & \textbf{Status} \\
	\hline
	Engage in product requirements analysis             & July 16          & 3 weeks           & Completed       \\
	Django backend web development                      & June 9          & 5 weeks           & In Progress     \\
	Front-end web development (HTML, CSS, JS)           & June 9          & 3 weeks           & Completed       \\
	Use case implementation and new feature development & June 9          & 3 weeks           & Completed       \\
	Web system testing \& refactoring                   & ---          & 2 weeks           & Planned         \\
	Web system deployment                               & ---          & 1 week            & Planned         \\
	\hline
\end{tabular}



\section{Work Done}

\begin{itemize}
    \item \textbf{Product Requirements Analysis:} I actively participated in the analysis of product requirements, including understanding the project's scope and identifying user needs and functional and non-functional requirements.
    
    \item \textbf{Django Backend Web Development:} I dedicated a significant portion of my time to Django backend web development. I worked on building robust and scalable backend components that form the core of the web application.
    
    
    \item \textbf{Front-end Web Development:} I delved into front-end web development using technologies like HTML, CSS, and JavaScript. I created user interfaces that are intuitive, visually appealing, and responsive.
    
    \item \textbf{Use Case Implementation and New Feature Development:} I collaborated with my supervisor to implement various use cases and introduce new features to the system. One of the key features we worked on was a service that allows employees to request loan. We meticulously designed Django models and developed 10 use cases to support this feature:
    
    \begin{itemize}[label=--]
        \item View Employee Loan
        \item View Employees Loans
        \item View Loan Requests
        \item Create Loan Request
        \item View Loan Request
        \item Delete Loan Request
        \item Update Loan Request
    \end{itemize}
    
\end{itemize}

\section{Upcoming Tasks}

\begin{itemize}
    \item \textbf{Web System Testing \& Refactoring:} In the upcoming weeks, my focus will be on testing the web system thoroughly. I will conduct unit tests, integration tests, and end-to-end tests to ensure the system's functionality, reliability, and security. Additionally, I will work on refactoring the codebase to improve its maintainability and readability.
    
    \item \textbf{Web System Deployment:} As we near the end of the training period, I will be involved in the process of deploying the web system. This will involve setting up servers, configuring databases, and ensuring a smooth transition from the development to the production environment.
    
\end{itemize}

\section{Conclusion}
Overall, this summer training has provided me with valuable learning opportunities. Engaging in real-world software engineering tasks, from requirements analysis to deployment, has given me a deeper understanding of the engineering process and the workflow involved in bringing an idea to life. I look forward to the challenges and learning experiences that the coming weeks will bring.

\end{document}

\href{https://github.com/ammar-faifi/latex}{LaTeX source here}


\end{document}
