\documentclass{article}

\usepackage[a4paper, margin=15mm]{geometry}
\usepackage{amsmath}
\usepackage{amsthm}
\usepackage{amssymb}
\usepackage{graphicx}
\usepackage{enumitem}
\usepackage{xcolor}

\usepackage{hyperref}
\hypersetup{
    colorlinks=true,
    linkcolor=blue,
    filecolor=magenta,
    urlcolor=blue,
}


\def\c#1{\texttt{#1}}

\title{Progress Report 1 - Summer Training (ICS399)}
\author{Alfaifi, Ammar - 201855360}
\date{}

\begin{document}

\maketitle

\section{Schedule}

\begin{table}[h]
  \centering
  \caption{Activities by Week}
  \begin{tabular}{|c|p{10cm}|}
    \hline
    \textbf{Week \#} & \textbf{Activities}       \\
    \hline
    1                & \begin{itemize}
      \item Familiarizing trainees with the company organization, product, tools, and frameworks.
      \item Introduction to project management practices.
      \item Assigning tasks to each trainee.
      \item Understanding the overall system architecture.
      \item Reviewing and updating the system requirement specifications (SRS) for the web system.
      \item Testing the web system and evaluating user experience.
    \end{itemize} \\
    \hline
    2 - 3            & \begin{itemize}
      \item Learning the Django framework.
      \item Immersion into the Mawqoot web architecture and existing codebase.
      \item Active involvement in product requirements analysis.
    \end{itemize} \\
    \hline
  \end{tabular}
\end{table}

\section{Accomplished work}

During the first week, the focus was on introducing trainees to various aspects of the company's organization, product, tools, and frameworks. Additionally,  we were familiarized with project management practices. To ensure a smooth start, specific tasks were assigned to each trainee, allowing them to actively participate in the learning process. One of the key objectives was to gain a thorough understanding of the general system architecture. This involved revising the system requirement specifications (SRS) for the web system, as well as conducting rigorous testing to evaluate the web system's functionality and user experience. Through these activities, I gained valuable experience in system analysis, testing techniques, and enhancing user satisfaction.
\newline

Moving into the second week, the focus shifted towards mastering the Django framework, which involved a hands-on approach. I delved into the Mawqoot web architecture, gaining practical experience by actively engaging with the product requirements analysis. This provided me with an opportunity to apply my newly acquired knowledge and techniques in a real-world context. By immersing in the project's requirements, I not only expanded my technical skills but also developed a deeper understanding of the interplay between system design and user needs.
\newline

Throughout these weeks, I accomplished a range of tasks, including reading the SRS for 12 django apps and doing black-box testing for them reporting and inconsistensies in the UI or faults and failurs. I did the same with 8 Flutter use cases. The work was use-case oriented. Where each trainee was assigned a number of them to read their SRS and test them in the web or mobile UIs. We revised the SRS, tested the web system, and improved the user experience. The approach taken in these activities was hands-on and practical, allowing me to actively participate and learn through real-world scenarios. As a result, I gained valuable techniques and experience in system analysis, testing, framework utilization, and aligning product requirements with user needs.


\section{Coming Plan}

For the coming weeks I'll focused on working more on real-world problems. Since I've signed the NDA contract, for next week I'll have a look at the source code to understand its structure as well as implementing my assigned task. Here is a brief of comming weeks' work

\begin{table}[h]
  \centering
  \caption{Activities by Week of coming weeks}

  \begin{tabular}{|c|p{10cm}|}
    \hline
    \textbf{Week \#} & \textbf{Activities}           \\
    \hline
    4 - 6            & \begin{itemize}
      \item Engage in product requirements analysis
      \item Django backend web development
      \item Engage in designing Django Rest API
      \item Front-end web development (HTML, CSS, JavaScript)
    \end{itemize} \\
    \hline
    7 - 8            & \begin{itemize}
      \item Web system testing \& refactoring
      \item Engage in web system deployment
    \end{itemize} \\
    \hline
  \end{tabular}
\end{table}

During the coming period, my work outlook will primarily revolve around engaging in product requirements analysis, Django backend web development, and designing Django Rest APIs. This entails collaborating with stakeholders to understand their needs and translating them into functional requirements. I will be utilizing Django, a popular web framework, to develop the backend of web applications, ensuring efficient data handling and seamless functionality. Additionally, I will be focusing on designing and implementing Django Rest APIs, which will enable smooth communication between the backend and frontend components of the web system.
\newline

In parallel, I will be involved in front-end web development, utilizing HTML, CSS, and JavaScript to create the user interface and enhance the overall user experience. This aspect of my work will involve creating visually appealing and responsive web pages that interact seamlessly with the backend functionality.
\newline

As the period progresses, my attention will shift towards web system testing and refactoring. This involves thoroughly testing the developed web system to identify and address any issues or bugs. Additionally, I will engage in refactoring, which involves restructuring the codebase to improve its readability, maintainability, and performance.


\href{https://github.com/ammar-faifi/latex}{LaTeX source here}


\end{document}
