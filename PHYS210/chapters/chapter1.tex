\chapter{Series} \label{ch:series}
        \section{Infinite Series}

            \begin{itemize}
                \item A \textit{sequence} is simply a set of quantities, one for each n, represented with $a_n$. 
                \item A \textit{series} is an indicated sum of such represented with $'\sum'$ symbol.
                \item In general \textbf{infinite series} can be written as $ \Rightarrow  a_1 + a_2 + a_3 + \dots + a_n + \dots $
                \item The geomtric series are written as $ \Rightarrow a+ar+ar^2+\dots+ar^{n-1} + \dots$
                \item or in short notation: $ \Rightarrow \sum_{n=1}^{\infty} a_n $
                \item The sum first of first: \textit{n} term is \colorbox{c1}{$ S_n=\frac{a(1-r^n)}{1-r} $}
                \begin{quote}
                    $\bigstar$ The sum of an infinite series is the limit of the sum of $n$ terms as $n\to\infty$.
                    The \textit{sum of series} is \colorbox{c1}{$ S = \lim_{n \to \infty} S_n=\frac{a}{1-r}$}
                \end{quote}
                
                \begin{quote}
                    $\bigstar$ The geomtric series have a finite sum if and only if $|r|<1$.\\
                    Then the series is called \textit{convergent}, Otherwise it is called \textit{divergent}.
                \end{quote}
                \item the \textit{remainder} (or the remainder after $n$ terms) is \colorbox{c1}{$R_n=S-S_n$}.\\
                Thus, \colorbox{c1}{$\lim_{n\to\infty}\, R_n = 0$.}
                % \coloredeq{R_n=S-S_n \\ \lim_{n\to\infty}\, R_n = 0}.
                
            \end{itemize}


        \section{Series Tests}

            \begin{itemize}
                \item If the terms of an infinite series do not tend to zero
                (that is, if a $\lim_{n\to\infty}\,$), the series diverges. 
                If $\lim_{n\to\infty}\, a_n=0$, we must test further (Except the alternating series).
                \item An alternating series is an example of a \textit{conditional series}.
                \item Its positive or negative terms (alone) \textit{diverges}, thus we can control 
                to which number (the sum, S) such series approaches.
                \item Physically, We cannot stop at some point and say that the rest of the series is 
                negligible as we could in the bouncing ball problem in Section 1.
                \item But if we specify the order in which the charges are to be placed, then the sum 
                $S$ of the series is determined ($S$ is probably different from $F$ in (8.1) unless 
                the charges are placed alternately).
            \end{itemize}
                
            \paragraph{$\bigstar$} 
                The convergence or divergence of a series is not affected by multiplying 
                every term of the series by the same nonzero constant. Neither is it affected by 
                changing a finite number of terms (for example, omitting the first few terms).

            \paragraph{$\bigstar$}
                Two \textit{convergent} series $\sum_{n=1}^{\infty}\, a_n$ and $\sum_{n=1}^{\infty}\, b_n$
                may be added (or subtracted) term by term ($a_n+b_n$).
                The resulting series is \textit{convergent}, and its sum is obtained
                by adding (subtracting) the sums of the two given series.
            
            \paragraph{$\bigstar$}
                The terms of an \textit{absolutely convergent series} may be rearranged in 
                any order without affecting either the \textit{convergence} or the \textit{sum}. This is not 
                true of \textit{conditionally convergent series} as we have seen in Section 8.

        \section{Power Series}
            It is distiguished by having a varible $x^n$ in its terms multiply by a constant.\\
            By definition: \coloredeq{eq:power}{\sum_{n=0}^{\infty}\, a_n(x-a)=a_0+a_1(x-a)^1+a_2(x-a)^2+\dots}.
            \begin{itemize}
                \item The \textit{radius of convergence} $R$ of Eq:\ref{eq:power} depends on $x$ values .
                \item We find $R$ by the \textit{ratio test} so that $L<1$.
            \end{itemize}

        \paragraph{$\bigstar$}
            We see then that a power series (within its interval of convergence) defines a 
            function of $x$, namely $S(x)$.

        \paragraph{Theorems}
            \begin{enumerate}
                \item A power series may be differentiated or integrated term by term; 
                the resulting series converges to the derivative or integral of the function 
                represented by the original series \textit{within} the same interval of convergence as the 
                original series (that is, not necessarily at the endpoints of the interval).
                \item Two power series may be added, subtracted, or multiplied; the resultant series 
                converges at least in the common interval of convergence. You may divide two series if 
                the denominator series is not zero at $x = 0$, or if it is and the zero is canceled by 
                the numerator [as, for example, in $(\sin{x})/x$; see (13.1)]. The resulting series will 
                have \textit{some} interval of convergence (which can be found by the ratio test or more simply 
                by complex variable theory—see Chapter 2, Section 7).
                \item One series may be substituted in another provided that the values of the substituted series 
                are in the interval of convergence of the other series.
                \item The power series of a function is unique, that is, there is just one power series of the 
                form $\sum_{n=0}^{\infty}\, a_nx^n$ which converges to a given function.
            \end{enumerate}

            \paragraph{$\bigstar$}
                Expanding functions can be done using \textit{Taylor series}
                \coloredeq{eq:taylor}{f(x)=\sum_{n=0}^{\infty}\, \frac{f^{(n)}(a)}{n!}(x-a)^n}
            
            \begin{itemize}
                \item \textit{Maclaurin series} for $f(x)$ is a special case of Eq\eqref{eq:taylor} where $a=0$.
            \end{itemize}