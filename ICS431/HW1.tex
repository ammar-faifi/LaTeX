\documentclass{article}

\usepackage[a4paper, margin=15mm]{geometry}
\usepackage{amsmath}
\usepackage{amsthm}
\usepackage{amssymb}
\usepackage{graphicx}


\def\code#1{\texttt{#1}}

\title{Homework 1 - Operating Systems (ICS431)}
\author{Alfaifi, Ammar}
\date{}

\begin{document}

\maketitle

\section{Linux Terminal}%
\subsection{\code{top}}%

It shows live current running processes

\begin{figure}[ht]
  \centering
  \includegraphics[width=0.7\textwidth]{~/Screenshots/Screenshot 2023-02-13 at 23.54.25.png}
\end{figure}


\subsection{\code{htop}}%

Similar to \code{top} but with very detailed info about the processes as well as
the RAM and CPU usages

\begin{figure}[ht]
  \centering
  \includegraphics[width=0.7\textwidth]{~/Screenshots/Screenshot 2023-02-13 at 23.54.33.png}
\end{figure}

\newpage

\subsection{\code{ps}}%
Show a set of the current running processes, with default parameters

\begin{figure}[ht]
  \centering
  \includegraphics[width=0.5\textwidth]{~/Screenshots/Screenshot 2023-02-16 at 11.50.37.png}
\end{figure}


\subsection{\code{ps -A | less}}%
\code{ps -A} shows all processes, the pipe operator ($|$) will redirect stdout of the left
command into the stdin of the right command. Here \code{less} command shows the output as
scrollable list.

\begin{figure}[ht]
  \centering
  \includegraphics[width=0.5\textwidth]{/Users/ammar-imac/Screenshots/Screenshot 2023-02-16 at 11.56.35.png}
\end{figure}


\subsection{\code{ps -A | grep}}%
As above but it use command \code{grep} for filtering
\begin{figure}[ht]
  \centering
  \includegraphics[width=0.5\textwidth]{/Users/ammar-imac/Screenshots/Screenshot 2023-02-16 at 11.59.00.png}
\end{figure}


\newpage


\subsection{\code{pstree}}%
Like \code{ps}, but in  tree-like format.
\begin{figure}[ht]
  \centering
  \includegraphics[width=0.5\textwidth]{/Users/ammar-imac/Screenshots/Screenshot 2023-02-16 at 12.19.27.png}
\end{figure}

\subsection{\code{kill}}%
Like \code{ps}, but in  tree-like format.
\begin{figure}[ht]
  \centering
  \includegraphics[width=0.5\textwidth]{/Users/ammar-imac/Screenshots/Screenshot 2023-02-16 at 12.41.57.png}
\end{figure}

\subsection{\code{pgrep}}%
Like \code{ps}, but in  tree-like format.
\begin{figure}[ht]
  \centering
  \includegraphics[width=0.5\textwidth]{/Users/ammar-imac/Screenshots/Screenshot 2023-02-16 at 12.42.38.png}
\end{figure}

\newpage


\subsection{\code{pkill}}%
Using power of filtering in \code{grep}, and perform \code{kill} for all matched pattern.
\begin{figure}[ht]
  \centering
  \includegraphics[width=0.5\textwidth]{/Users/ammar-imac/Screenshots/Screenshot 2023-02-16 at 12.42.46.png}
\end{figure}

\subsection{\code{killall}}%
Send a kill signal to all processes by command name.
\begin{figure}[ht]
  \centering
  \includegraphics[width=0.5\textwidth]{/Users/ammar-imac/Screenshots/Screenshot 2023-02-16 at 12.42.59.png}
\end{figure}

\subsection{\code{pstree}}%
Change the priority value of running processes.
\begin{figure}[ht]
  \centering
  \includegraphics[width=0.5\textwidth]{/Users/ammar-imac/Screenshots/Screenshot 2023-02-16 at 12.57.53.png}
\end{figure}


\newpage


\subsection{\code{xkill}}%
forcing the X server to close connections to clients.
\begin{figure}[ht]
  \centering
  \includegraphics[width=0.5\textwidth]{/Users/ammar-imac/Screenshots/Screenshot 2023-02-16 at 12.42.46.png}
\end{figure}

\subsection{\code{nohup}}%
To run a command that cannot be affected by hangup singal.
\begin{figure}[ht]
  \centering
  \includegraphics[width=0.5\textwidth]{/Users/ammar-imac/Screenshots/Screenshot 2023-02-16 at 12.42.59.png}
\end{figure}

\subsection{\code{ls}}%
List all the current working directory contents.
\begin{figure}[ht]
  \centering
  \includegraphics[width=0.5\textwidth]{/Users/ammar-imac/Screenshots/Screenshot 2023-02-16 at 12.57.53.png}
\end{figure}


\newpage


\subsection{\code{ls -R}}%
forcing the X server to close connections to clients.
\begin{figure}[ht]
  \centering
  \includegraphics[width=0.5\textwidth]{/Users/ammar-imac/Screenshots/Screenshot 2023-02-16 at 13.23.20.png}
\end{figure}

\subsection{\code{ls -a}}%
Do not ignore contents with name starting with dot (.example).
\begin{figure}[ht]
  \centering
  \includegraphics[width=0.5\textwidth]{/Users/ammar-imac/Screenshots/Screenshot 2023-02-16 at 13.23.27.png}
\end{figure}

\subsection{\code{ls -al}}%
In recursive mode, meaning list all sub directories
\begin{figure}[ht]
  \centering
  \includegraphics[width=0.5\textwidth]{/Users/ammar-imac/Screenshots/Screenshot 2023-02-16 at 13.23.41.png}
\end{figure}


\newpage


\subsection{\code{cat > filename}}%
Create a file with the name \code{filename}, and anything typed in terminal will be part of the file
until termination by Ctrl+X.
\begin{figure}[ht]
  \centering
  \includegraphics[width=0.5\textwidth]{/Users/ammar-imac/Screenshots/Screenshot 2023-02-16 at 16.09.44.png}
\end{figure}

\subsection{\code{cat filename}}%
Prints a file named \code{filename} to the stdout.
\begin{figure}[ht]
  \centering
  \includegraphics[width=0.5\textwidth]{/Users/ammar-imac/Screenshots/Screenshot 2023-02-16 at 16.09.52.png}
\end{figure}

\subsection{\code{cat file1 file2 > file3}}%
Take content of both \code{file1} \code{file2} into new file named \code{file3}
\begin{figure}[ht]
  \centering
  \includegraphics[width=0.5\textwidth]{/Users/ammar-imac/Screenshots/Screenshot 2023-02-16 at 16.11.39.png}
\end{figure}

\newpage


\subsection{\code{mv file new/file/path'}}%
It can be used to move file to another dir, if it's the same dir, it rename the file.
\begin{figure}[ht]
  \centering
  \includegraphics[width=0.5\textwidth]{/Users/ammar-imac/Screenshots/Screenshot 2023-02-16 at 16.22.19.png}
\end{figure}

\subsection{\code{mv filename new\_file\_name}}%
Moved the file to the same directory, meaning it got renamed.
\begin{figure}[ht]
  \centering
  \includegraphics[width=0.5\textwidth]{/Users/ammar-imac/Screenshots/Screenshot 2023-02-16 at 16.22.44.png}
\end{figure}

\subsection{\code{sudo}}%
Super user do; run a command in the root user permission.
\begin{figure}[ht]
  \centering
  \includegraphics[width=0.5\textwidth]{/Users/ammar-imac/Screenshots/Screenshot 2023-02-16 at 16.22.55.png}
\end{figure}

\newpage

\subsection{\code{rm filename}}%
Remove the file name \code{filename} from existence.
\begin{figure}[ht]
  \centering
  \includegraphics[width=0.5\textwidth]{/Users/ammar-imac/Screenshots/Screenshot 2023-02-16 at 16.31.14.png}
\end{figure}

\subsection{\code{man ls}}%
Remove the file name \code{filename} from existence.
\begin{figure}[ht]
  \centering
  \includegraphics[width=0.5\textwidth]{/Users/ammar-imac/Screenshots/Screenshot 2023-02-20 at 16.41.35.png}
\end{figure}

\subsection{\code{netstat}}%
Print  network  connections, routing tables, interface statistics, masquerade connections, and multicast memberships
\begin{figure}[ht]
  \centering
  \includegraphics[width=0.5\textwidth]{/Users/ammar-imac/Screenshots/Screenshot 2023-02-20 at 16.51.41.png}
\end{figure}

\newpage

\subsection{\code{clear}}%
Clear all outputs

\subsection{\code{mkdir dir-name}}%
Create a new directory
\begin{figure}[ht]
  \centering
  \includegraphics[width=0.5\textwidth]{/Users/ammar-imac/Screenshots/Screenshot 2023-02-20 at 17.04.57.png}
\end{figure}


\subsection{\code{mv old\_file new\_file}}%
We knew the \code{mv} moves file from a path to a new one, but if it's on the same path, it's like renaming
the file.
\begin{figure}[ht]
  \centering
  \includegraphics[width=0.5\textwidth]{/Users/ammar-imac/Screenshots/Screenshot 2023-02-20 at 17.08.55.png}
\end{figure}


\subsection{\code{pr -x}}%
This is not a valid argument.

\subsection{\code{pr -h}}%
It helps to prepare file for printing it, the option \code{-h} is
used to a centered HEADER instead of filename in page header.
\begin{figure}[ht]
  \centering
  \includegraphics[width=0.5\textwidth]{/Users/ammar-imac/Screenshots/Screenshot 2023-02-20 at 17.17.37.png}
\end{figure}

\newpage

\subsection{\code{pr -n}}%
Same as above, but adds line numbering.
\begin{figure}[ht]
  \centering
  \includegraphics[width=0.5\textwidth]{/Users/ammar-imac/Screenshots/Screenshot 2023-02-20 at 17.22.39.png}
\end{figure}


\subsection{\code{vmstat}}%
It reports  information  about processes, memory, paging, block IO, traps,
disks and cpu activity.
\begin{figure}[ht]
  \centering
  \includegraphics[width=0.5\textwidth]{/Users/ammar-imac/Screenshots/Screenshot 2023-02-20 at 17.26.20.png}
\end{figure}

\subsection{\code{df}}%
It reports file system disk space usage
\begin{figure}[ht]
  \centering
  \includegraphics[width=0.5\textwidth]{/Users/ammar-imac/Screenshots/Screenshot 2023-02-20 at 17.26.27.png}
\end{figure}

\newpage

\subsection{\code{vi filename}}%
Command for the \code{vim} editor. It opens the file called \code{filename} or creates it.
\begin{figure}[ht]
  \centering
  \includegraphics[width=0.5\textwidth]{/Users/ammar-imac/Screenshots/Screenshot 2023-02-20 at 17.26.46.png}
\end{figure}

\subsection{\code{traceroute}}%
tracks the route packets taken from an IP network on their way to a given host.
\begin{figure}[ht]
  \centering
  \includegraphics[width=0.5\textwidth]{/Users/ammar-imac/Screenshots/Screenshot 2023-02-20 at 17.39.08.png}
\end{figure}

\subsection{\code{cp file new\_file}}%
It copies a content of file \cdoe{file} to a new one called \code{new\_file}.
\begin{figure}[ht]
  \centering
  \includegraphics[width=0.5\textwidth]{/Users/ammar-imac/Screenshots/Screenshot 2023-02-20 at 17.39.37.png}
\end{figure}

\newpage

\subsection{\code{mv dir new\_die}}%
It moves directories into another one.
\begin{figure}[ht]
  \centering
  \includegraphics[width=0.5\textwidth]{/Users/ammar-imac/Screenshots/Screenshot 2023-02-20 at 17.22.39.png}
\end{figure}


\subsection{\code{history}}%
It shows  previous executed commands.
\begin{figure}[ht]
  \centering
  \includegraphics[width=0.5\textwidth]{/Users/ammar-imac/Screenshots/Screenshot 2023-02-20 at 17.40.11.png}
\end{figure}



\end{document}
