\documentclass{report}

\usepackage[a4paper, lmargin=20mm, rmargin=20mm]{geometry}
\usepackage{amsmath}
\usepackage{amsthm}
\usepackage{amssymb}
\usepackage{graphicx}

\def\code#1{\texttt{#1}}

\title{ICS431 Ch. 6 \& 7}
\author{Alfaifi, Ammar}
\date{}

\begin{document}

\maketitle


\chapter{Synchronization Tools}%
\label{sec:synchronization tool}

\section{Background}

\begin{itemize}
	\item A \textbf{cooperating process} is one that can affect or be affected by other processes
	      executing in the system.
	\item Cooperating processes can either directly share a logical address space or be allowed
	      to share data through message passing.
	\item Concurrent access to shared data may result in data inconsistency, so synchronization is a must!
	\item Recall the \textbf{unbound buffer} places no practical limit on the size of the buffer. The consumer
	      may have to wait for new items, but the producer can always add new items.
	\item For the \textbf{bounded buffer}, it assumes a fixed buffer size. The consumer must wait if the buffer is
	      empty, and the producer must wait if it's full.
	\item We consider the code of bounded buffer from Section 3.5, we add a new variable \code{count}, to
	      track the number of items in the buffer.
	\item As you see in the consumer's and producer's codes, page 274, the producer increment \code{count} and
	      the producer tries to decrement it each time. When running concurrently, there are clashing!
	\item When several processes access and manipulate the same data concurrently and the
	      outcome of the execution depends on the particulate order in which the access takes place is called
	      \textbf{race condition}.
	\item Since we are interleaving the lower-level statements of both codes (but the order withing each
	      high-level is preserved), both \code{count++} and \code{count--} are executed using the same old value,
	      resulting wrong result; this is a race condition.
\end{itemize}

\section{The Critical-Section Problem}%

\begin{itemize}
	\item Consider a system with $ n $ processes $ {P_0, P_1, \dots, P_{n-1}} $. Each process ha a segment of code
	      is a \textbf{critical section}, meaning a process can modify a data that's shared with at least one
	      other process.
	\item The section of code implementing this request is called \textbf{entry section}. There can be also
	      \textbf{exit section}.
	\item A solution to he critical-section problem must satisfy the following
	      \begin{enumerate}
		      \item \textbf{Mutual exclusion} if process $P_i$ is executing it its critical section, no other processes
		            can be executing in their critical sections.
		      \item \textbf{Progress} if no process is executing in its critical section and some processes want to
		            enter their critical sections, then only those aren't executing in their remainder section can
		            decide which will enter its critical section next.
		      \item \textbf{Bounded waiting} a bound (limit) on the number of times that other processes are allowed
		            to enter their critical section after a process made a request to enter its critical section
		            and that request.
	      \end{enumerate}
	\item Let \code{next\_available\_pid} be the pid to be assigned to a process after calling \code{fork()}. If
	      \code{fork()} is called twice these will lead to a race condition in a kernel data structure.
	\item This problem can be avoided in single-core environment, by disabling interrupts while a modifying
	      a shared variable. But this inefficiency solution, and not feasible in multiprocessor environment.
	\item Two approaches are used to handle this problem:
	      \begin{enumerate}
		      \item \textbf{Preemptive kernels} allows a process to be preempted while it is running in kernel mode.
		      \item \textbf{Nonpreemptive kernels} does not allow a process running in kernel mode to be
		            preempted; a kernel-mode process will run until it exits kernel mode, blocks, or yields control of
		            CPU back.
	      \end{enumerate}
	\item Obviously, nonpreemptive doesn't have the race condition problem on kernel data structure,
	      since only one process at time can run.
\end{itemize}

\section{Peterson's Solution}%

\begin{itemize}
	\item This is a software-based solution to the critical-section problem.
	\item It requires the two processes to share two data items as, written in \code{C++}
	      \begin{verbatim}
            int turn;
            boolean flags[2];
\end{verbatim}
	\item \code{turn} indicates whose turn it is to enter its critical section.
	      if \code{turn == i}, then process $ P_{i} $ is allowed to execute its critical section.
	\item \code{flag} array indicates if a process is ready to enter its critical section;
	      if \code{flag[i]} is \code{true}, $ P_i $ is ready to enter its critical section.
	\item The process $ P_i $	code should be
	      \begin{verbatim}
						While (true) {
							flag[i] = true;
							turn = j;
							while (flag[j] && turn == j)
								;

							/* Here is the critical section */

							flag[i] = false;

							/* remainder section */
					}
				\end{verbatim}
	\item A downside of the solution is that, it is affected by compilation process where reordering
	      for the read and write operation may happen.
\end{itemize}

\section{Hardware Support for Synchronization}%
\begin{enumerate}
	\item \textbf{Memory Barriers}
	      \begin{itemize}
		      \item We saw that system may reorder instructions, leading to unreliable data states.
		      \item \textbf{memory model} is how a computer determines what memory should provide to an application
		            program.
		            \begin{enumerate}
			            \item \textit{Strongly ordered}: A memory modification in one process is visible immediately to all other processes.
			            \item \textit{Weakly ordered}: Modifications to memory on one processor may not be visible to others.
		            \end{enumerate}
		      \item Memory models vary by process\textit{or}, so kernel developers cannot make assumptions if the visibility of
		            modifications to memory on a shared-memory multiprocessor.
		      \item computer architectures provide instructions that can \textit{force} any changes in memory to be propagated
		            to all other possessors, they're called \textbf{memory barriers}
		      \item The idea is that when a memory barrier instruction is performed, the system ensures all \code{load} \&
		            \code{store} are completed before any coming \code{load} and \code{store} are executed.
		      \item Consider this code
		            \begin{verbatim}
x = 100;
memory_barrier();
flag = true;
\end{verbatim}
		            we ensure the assignment to \code{x} occurs before the assignment to \code{flag}.
	      \end{itemize}
	\item \textbf{Hardware instructions}
	      \begin{itemize}
		      \item Modern computer systems provide special hardware instructions that allow us either to test and modify the content of a word or
		            to swap the contents of two words \textbf{atomically}, as one uninterruptible unit.
		      \item  Then we use these special instructions to solve the critical-section problem.
		      \item In abstract, consider this code
		            \begin{verbatim}
boolean test_and_set(boolean *target) {
    boolean rv = *target;
    *target = true;

    return rv;
}
\end{verbatim}

		      \item This instruction is a low-level atomic operation used in operating systems and multi-threaded programming. It is used to set a flag or a lock on a shared resource while checking if it was previously set. The instruction performs the following steps atomically:

		            \begin{enumerate}
			            \item Read the current value of the flag or lock.
			            \item Set the flag or lock to the desired value.
			            \item Return the previous value of the flag or lock.
		            \end{enumerate}

		      \item This operation is used to implement synchronization primitives such as mutexes, semaphores, and spin locks. It ensures that multiple threads or processes do not access the shared resource simultaneously, preventing race conditions and other synchronization issues.


		      \item The main characteristic of this instruction is that it is executed atomically;
		            if two \code{test\_and\_set()} instructions are executed
		            simultaneously, they will be executed sequentially in arbitrary order.
		      \item If the machine support this instruction, then we can implement \textit{mutual exclusion} by declaring a boolean
		            variable \code{lock}, initialized to \code{false}. The structure of process $ P_{i} $ becomes
		            \begin{verbatim}
do {
    while (test_and_set(&lock))
        ;

        /* critical section */

        lock = false;

        /* remainder section */
} while(true);
\end{verbatim}

		      \item \code{compare\_and\_swap()} (CAS) is another low-level atomic operation
		            It is used to update the value of a memory location atomically, while
		            also checking if the current value of the memory location matches an expected value.

		      \item It takes three arguments: a pointer to the memory location
		            to be updated, the expected value of that memory location, and the new value to be written
		            to that memory location. The operation will only update the memory location if its current value
		            matches the expected value.

	      \end{itemize}
	\item \textbf{Atomic variables}


\end{enumerate}

\end{document}
