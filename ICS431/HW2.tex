\documentclass{article}

\usepackage[a4paper, margin=15mm]{geometry}
\usepackage{amsmath}
\usepackage{amsthm}
\usepackage{amssymb}
\usepackage{graphicx}
\usepackage{enumitem}
\usepackage{listings}
\usepackage{xcolor}

\definecolor{codeblue}{RGB}{34, 112, 177}
\definecolor{codegreen}{RGB}{28, 172, 120}
\definecolor{codegray}{RGB}{100, 100, 100}
\definecolor{codepurple}{RGB}{139, 0, 255}
\definecolor{backgray}{RGB}{245, 245, 245}

\lstdefinestyle{cstyle}{
    language=C,
    backgroundcolor=\color{backgray},
    basicstyle=\footnotesize\ttfamily,
    keywordstyle=\color{codepurple},
    commentstyle=\color{gray},
    stringstyle=\color{codegreen},
    numbers=left,
    numberstyle=\tiny\color{codegray},
    stepnumber=1,
    numbersep=10pt,
    tabsize=4,
    showspaces=false,
    showstringspaces=false
}

\def\c#1{\texttt{#1}}

\title{Homework 2 - Operating Systems (ICS431)}
\author{Alfaifi, Ammar}
\date{}

\begin{document}

\maketitle

\section{Commands}%
\subsection{\c{ps}}%

This shows processes status. without any options, it will show you a list of processes
running in your current terminal session.

\begin{figure}[ht]
	\centering
	\includegraphics[width=0.6\textwidth]{/Users/ammar-imac/Screenshots/Screenshot 2023-03-18 at 16.50.44.png}
\end{figure}


\subsection{\c{ps -e}}%

show all processes, not just those related to the current terminal session

\begin{figure}[ht]
	\centering
	\includegraphics[width=0.6\textwidth]{/Users/ammar-imac/Screenshots/Screenshot 2023-03-18 at 18.47.45.png }
\end{figure}


\subsection{\c{ps -u}}%

will show only the processes started by the current user.

\begin{figure}[ht]
	\centering
	\includegraphics[width=0.6\textwidth]{/Users/ammar-imac/Screenshots/Screenshot 2023-03-18 at 18.47.59.png }
\end{figure}

\newpage

\subsection{\c{ps -el}}%
command will show the complete list of processes in a long format. It will include the process ID, parent process ID, user ID, group ID, virtual memory size, resident set size, CPU usage, start time, terminal, and command.

\begin{figure}[ht]
	\centering
	\includegraphics[width=0.9\textwidth]{/Users/ammar-imac/Screenshots/Screenshot 2023-03-18 at 18.48.12.png }
\end{figure}

\section{Zombie Processes}%
This code is to simulate a zombie process.

\begin{lstlisting}[style=cstyle]
#include <stdio.h>
#include <stdlib.h>
#include <sys/wait.h>
#include <unistd.h>

int main() {
  pid_t pid;
  pid = fork();

  if (pid < 0) {
    fprintf(stderr, "Fork failed.\n");
    exit(1);
  } else if (pid == 0) {
    printf("This is the child process.\n");
    exit(0);
  } else {
    printf("This is the parent process. Child's PID %d\n", pid);
    sleep(10);
    waitpid(pid, NULL, 0); // wait for child process to exit
    printf("Parent process exiting.\n");
    exit(0);
  }

  return 0;
}
\end{lstlisting}


\newpage

From the follwing screenshots, the process with PID of 29296 is the \c{bash} itself.
The running C program has PID of 33017, its parent is the \c{bash}. When \c{fork()} is called,
a duplicate process from the parent (PID:33017) is created with PID of 33020. This child's PID is returned
in the parent process, as seen in the first line in the second figure. When the child is finished but
the parent is not, the child's process (33020) becomes a zombie process (see the status \c{Z}).

\begin{figure}[ht]
	\centering
	\includegraphics[width=0.8\textwidth]{/Users/ammar-imac/Screenshots/Screenshot 2023-03-18 at 21.00.13.png}
\end{figure}


\section{Multithreaded Program}%

The codes in C is
\begin{lstlisting}[style=cstyle]
#include <pthread.h>
#include <stdio.h>
#include <stdlib.h>

void *print_primes(void *arg) {
  int num = *(int *)arg;
  int i, j, flag;
  for (i = 2; i <= num; i++) {
    flag = 1;
    for (j = 2; j <= i / 2; j++) {
      if (i % j == 0) {
        flag = 0;
        break;
      }
    }
    if (flag == 1) {
      printf("%d ", i);
    }
  }
  printf("\n");
  pthread_exit(NULL);
}

int main(int argc, char *argv[]) {
  int num = atoi(argv[1]);

  pthread_t tid;
  pthread_attr_t attr;
  pthread_attr_init(&attr);

  pthread_create(&tid, &attr, print_primes, &num);
  pthread_join(tid, NULL);

  return 0;
}
\end{lstlisting}


\newpage
\paragraph{}
In the following screenshot, I ran the program with an input value of 100. Next I ran it with very large
number so I can watch its thread using \c{ps -lfL} command. I use \c{>} to pipe the \c{stdout} to
the virtual device \c{/dev/null} to discard any output, then I use \c{\&} to run it in background.

\paragraph{}
From the output of \c{ps}, there are to processes with same PID which indicated it's a multithreaded
process. We see also the value of the thread ID (LWP). Also the CPU utilization (C) is large in the thread
with value of 93.
\begin{figure}[ht]
	\centering
	\includegraphics[width=0.9\textwidth]{/Users/ammar-imac/Screenshots/Screenshot 2023-03-18 at 21.56.06.png }
\end{figure}



\section{Proc files}%
\begin{enumerate}
	\item By the commnad \c{cat cpuinfo}, it shows 4 CPU cores in my machine.
	      \begin{figure}[ht]
		      \centering
		      \includegraphics[width=0.6\textwidth]{/Users/ammar-imac/Screenshots/Screenshot 2023-03-18 at 22.25.50.png }
	      \end{figure}
	      \newpage
	\item From \c{cat meminfo}, I have 4002372 kB ($\sim$ 4 GB), this is true because I'm running VM in my Mac and I
	      specify 4 GB to the Ubuntu VM. And available of 287492 kB (287.49 MB).
	      \begin{figure}[ht]
		      \centering
		      \includegraphics[height=0.7\textwidth]{/Users/ammar-imac/Screenshots/Screenshot 2023-03-18 at 22.28.45.png }
	      \end{figure}
	\item From running \c{cat stat}, then from the line field \c{ctxt}, It had 5417758 context switches.
	      \begin{figure}[ht]
		      \centering
		      \includegraphics[width=0.7\textwidth]{/Users/ammar-imac/Screenshots/Screenshot 2023-03-18 at 22.44.12.png }
	      \end{figure}
	\item From the same output I see that my system had forked 42381 processes.
\end{enumerate}


\newpage
\section{Shell Program in C}%
\begin{lstlisting}[style=cstyle][style=cstyle]
#include <stdio.h>
#include <stdlib.h>
#include <string.h>
#include <sys/wait.h>
#include <unistd.h>

#define MAX_COMMAND_LENGTH 100
#define MAX_ARGUMENTS 10

int main() {
  char command[MAX_COMMAND_LENGTH];
  char *arguments[MAX_ARGUMENTS];
  pid_t pid;
  int status;
  printf("Enter commands (type 'exit' to quit): \n");
  while (1) {
    printf("201855360> ");
    fgets(command, MAX_COMMAND_LENGTH, stdin);
    // Remove newline character
    command[strlen(command) - 1] = '\0';
    if (strcmp(command, "exit") == 0) {
      printf("Exiting shell program...\n");
      break;
    }
    // Parse command-line arguments
    char *token;
    int i = 0;
    token = strtok(command, " ");
    while (token != NULL) {
      arguments[i++] = token;
      token = strtok(NULL, " ");
    }
    arguments[i] = NULL; // Set last element to NULL for execvp
    pid = fork();
    if (pid == -1) {
      perror("fork");
      exit(EXIT_FAILURE);
    } else if (pid == 0) {
      // Child process
      if (execvp(arguments[0], arguments) == -1) {
        perror("execvp");
        exit(EXIT_FAILURE);
      }
    } else {
      // Parent process
      if (waitpid(pid, &status, 0) == -1) {
        perror("waitpid");
        exit(EXIT_FAILURE);
      }
      printf("Terminated: ");
      if (WIFEXITED(status)) {
        printf("Normally\n");
        printf("Exit status: %d\n", WEXITSTATUS(status));
      } else if (WIFSIGNALED(status)) {
        printf("Due to signal\n");
        printf("Signal number: %d\n", WTERMSIG(status));
      }
    }
  }
  return 0;
}
\end{lstlisting}

\newpage
Images for running different commands

\begin{figure}[ht]
	\centering
	\includegraphics[width=1\textwidth]{/Users/ammar-imac/Screenshots/Screenshot 2023-03-18 at 23.14.05.png }
	\caption{A screenshot for the ran program.}
\end{figure}


\newpage
\section{N Threads}

\begin{lstlisting}[style=cstyle]
#include <pthread.h>
#include <stdio.h>
#include <stdlib.h>

#define N 100
#define RANGE 10

int A[N], B[N], C[N];

void *sum(void *arg) {
  int i = *(int *)arg;
  C[i] = A[i] + B[i];
  return NULL;
}

int main() {
  srandom(time(NULL)); // randomize seed

  // initialize A and B
  for (int i = 0; i < N; i++) {
    A[i] = random() % RANGE;
    B[i] = random() % RANGE;
  }

  // create N threads to compute C[i] = A[i] + B[i]
  pthread_t threads[N];
  for (int i = 0; i < N; i++) {
    pthread_create(&threads[i], NULL, sum, &i);
  }

  // wait for all threads to finish
  for (int i = 0; i < N; i++) {
    pthread_join(threads[i], NULL);
  }

  // print A, B, and C
  printf("A = [");
  for (int i = 0; i < N; i++) {
    printf("%d ", A[i]);
  }
  printf("]\n");

  printf("B = [");
  for (int i = 0; i < N; i++) {
    printf("%d ", B[i]);
  }
  printf("]\n");

  printf("C = [");
  for (int i = 0; i < N; i++) {
    printf("%d ", C[i]);
  }
  printf("]\n");

  return 0;
}
\end{lstlisting}


The following figure shows the output of executing the program.

\begin{figure}[ht]
	\centering
	\includegraphics[width=1\textwidth]{/Users/ammar-imac/Screenshots/Screenshot 2023-03-19 at 08.28.24.png}
	\caption{A screenshot for the ran program.}
\end{figure}


\newpage

\section{Theoretical Part}%
\begin{enumerate}
	\item When a kernel context-switches between processes, it saves the current process state, including program counter, registers, and stack pointer, into the current process's PCB. It then loads the saved state of the next process to be executed from its PCB into the CPU's registers and program counter. The kernel also updates the memory management unit to switch the virtual memory mappings to the next process's memory space.

	\item Ordinary pipes are more suitable in situations where a simple communication channel is required between two processes executing on the same machine. For example, a parent process might create a child process and want to pass data to the child's standard input. Named pipes are more suitable in situations where multiple processes need to communicate with each other, possibly across different machines. For example, in a distributed system, several processes might need to communicate over a network using a named pipe.

	\item Yes, it is possible to have concurrency without parallelism. Concurrency refers to the ability of multiple tasks to be executed simultaneously, while parallelism refers to the actual simultaneous execution of those tasks on multiple CPUs or cores. For example, a single-core CPU can execute multiple threads in a concurrent manner, but they will not execute in parallel.

	\item When a thread is created, it uses fewer resources than when a process is created because it shares the same memory space as the parent process. The resources used include a program counter, a stack, and registers. When a process is created, it requires its own memory space, including a separate virtual address space and memory allocation for code, data, and stack.

	\item
	      \begin{enumerate}
		      \item CPU utilization and response time can conflict in situations where a process with a high CPU utilization is causing other processes to wait for CPU time, resulting in longer response times.
		      \item I/O device utilization and CPU utilization can conflict in situations where a process with high I/O utilization is blocking the CPU from executing other processes, resulting in reduced CPU utilization.
	      \end{enumerate}

	\item
	      \begin{enumerate}
		      \item First-Come-First-Serve (FCFS) scheduling algorithm discriminates against short processes because long processes can monopolize the CPU, causing short processes to wait longer.
		      \item Round-Robin (RR) scheduling algorithm discriminates in favour of short processes because each process is given a time slice, and short processes can complete their work within a single time slice.
		      \item Multilevel feedback queues scheduling algorithm discriminates in favour of short processes because it assigns shorter time slices to processes with higher priority.
	      \end{enumerate}
\end{enumerate}





\end{document}
