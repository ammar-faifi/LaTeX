\documentclass{book}

\usepackage[a4paper, lmargin=20mm, rmargin=20mm]{geometry}
\usepackage{amsmath}
\usepackage{amsthm}
\usepackage{amssymb}
\usepackage{graphicx}
\usepackage{siunitx}

\title{PHYS430 - Thermal Physics}
\author{Alfaifi, Ammar}
\date{}

\begin{document}

\maketitle

\chapter{Energy in Thermal Physics}


\section{Thermal Equilibrium}%
\label{sec:thermal-equilibrium}

\begin{itemize}
  \item After two objects have been in contact long enough, we say that they are in \textbf{thermal equilibrium}.
  \item The time required for a system to come to thermal equilibrium is called the \textbf{relaxation time}.
  \item \textbf{Temperature} is a measure of the tendency of an object to spontaneously give up energy to its
        surroundings.
  \item The flow of energy is from the object with a higher temperature to the lower on.
  \item For low-density gas at constant pressure, the volume should go to \textit{zero} at
        approximately $-273^{\circ}$C. which defines the \textbf{absolute zero}, in the
        \textbf{absolute temperature scale}, in K (kelvin).
\end{itemize}

\section{The Ideal Gas}%
\label{sec:The Ideal Gas}
\begin{align}
  PV = nRT; \qquad R = \si{8.31}{J / mol . K}
\end{align}

\begin{itemize}
  \item A \textbf{mole} of molecules is Avogadro's number of them, \num{6.02d23}.
  \item Number of molecules is $N=n \times N_{A}$
  \item Ideal gas law becomes $PV = NkT$, where $k$ is Boltzmann's constant.
  \item The average transnational kinetic energy is $\bar{K}_{\text{trans}}= \frac{3}{2}kT$,
        where $kT = \frac{1}{40} \si{\electronvolt}$
\end{itemize}

\section{Equipartition of Energy}%
\label{sec:equi of energy}

\paragraph{Equipartition theorem} At a temperature $T$, the average energy of any
quadratic degree of freedom is $\frac{1}{2}kT$.

For a system of $N$ molecules, each with $f$ degree of freedom, and there are no other
(non-quadratic) temperature-dependent forms of energy, then its \textbf{total thermal energy} is

\begin{align}
U = Nf \frac{1}{2}kT
\end{align}
Note, This is the \textit{average} total thermal energy,
but for large $N$, fluctuations become negligible.


\section{Heat and Work}%
\label{sec:heat and work}

\begin{itemize}
  \item Total amount of energy in the universe never changes, \textbf{Conservation of energy}
  \item \textbf{Heat} any spontaneous flow of energy form on e object to another, caused by
        difference in temperature.
  \item \textbf{Work}, in thermodynamics, is any other transfer of energy into or out of a system.
  \item Work and heat refer to energy \textit{in transit}.
  \item The total energy in a system is determined, but not the work nor the heat, it's meaningless.
  \item We ask about how much heat \textit{entered} a system and how much work
        \textit{was done on} a system.
  \item $\Delta{U} = Q + W$
        is just a statement of the law of conservation of energy, but it's still called
        \textbf{first law of thermodynamics}.
\end{itemize}

 \section{Commpression Work}%
 \label{sec:Compression Work}

\begin{itemize}
 \item From classical mmechanics work is $W = \vec{F} \cdot d\vec{r}$
 \item  Consider compressing gas with a piston of area $A$ a distance $\Delta{x}$,
   the change in volume is $\Delta{V} = -A \Delta{x}$
 \item Volume change should be quasistatic, meaning very slow so that the pressure defined is uniform.
     then $W = P A \Delta{V}$, but $\Delta{x} = - \Delta{V}$; minus since the volume decreases.
 \item $W =- P A \Delta{V}$  -  quasistatic.
 \item If $P$ is not constant, $$W = - \int_{V_i}^{V_f} P(V) dV$$
  \item \textbf{isothermal compression} is slow thst the temperature doesn't raise.
  \item \textbf{adiabatic compression} is so fast that no heat escapes from the gas.
\end{itemize}


\chapter{The Second Law}

\section{Two-State system}%
\label{sec:two state}



\end{document}
