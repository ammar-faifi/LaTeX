\documentclass{book}

\usepackage[a4paper, lmargin=20mm, rmargin=20mm]{geometry}
\usepackage{amsmath}
\usepackage{amsthm}
\usepackage{amssymb}
\usepackage{graphicx}
\usepackage{siunitx}

\title{PHYS430 - Thermal Physics}
\author{Alfaifi, Ammar}
\date{}

\begin{document}

\maketitle

\chapter{Energy in Thermal Physics}


\section{Thermal Equilibrium}%
\label{sec:thermal-equilibrium}

\begin{itemize}
	\item After two objects have been in contact long enough, we say that they are in \textbf{thermal equilibrium}.
	\item The time required for a system to come to thermal equilibrium is called the \textbf{relaxation time}.
	\item \textbf{Temperature} is a measure of the tendency of an object to spontaneously give up energy to its
	      surroundings.
	\item The flow of energy is from the object with a higher temperature to the lower on.
	\item For low-density gas at constant pressure, the volume should go to \textit{zero} at
	      approximately $-273^{\circ}$C. which defines the \textbf{absolute zero}, in the
	      \textbf{absolute temperature scale}, in K (kelvin).
\end{itemize}

\section{The Ideal Gas}%
\label{sec:The Ideal Gas}
\begin{align}
	PV = nRT; \qquad R = \qty{8.31}{J / mol . K}
\end{align}

\begin{itemize}
	\item A \textbf{mole} of molecules is Avogadro's number of them, \num{6.02d23}.
	\item Number of molecules is $N=n \times N_{A}$
	\item Ideal gas law becomes $PV = NkT$, where $k$ is Boltzmann's constant.
	\item The average transnational kinetic energy is $\bar{K}_{\text{trans}}= \frac{3}{2}kT$,
	      where $kT = \frac{1}{40} \si{\electronvolt}$
\end{itemize}

\section{Equipartition of Energy}%
\label{sec:equi of energy}

\paragraph{Equipartition theorem} At a temperature $T$, the average energy of any
quadratic degree of freedom is $\frac{1}{2}kT$.

For a system of $N$ molecules, each with $f$ degree of freedom, and there are no other
(non-quadratic) temperature-dependent forms of energy, then its \textbf{total thermal energy} is

\begin{align}
	U = Nf \frac{1}{2}kT
\end{align}
Note, This is the \textit{average} total thermal energy,
but for large $N$, fluctuations become negligible.


\section{Heat and Work}%
\label{sec:heat and work}

\begin{itemize}
	\item Total amount of energy in the universe never changes, \textbf{Conservation of energy}
	\item \textbf{Heat} any spontaneous flow of energy form on e object to another, caused by
	      difference in temperature.
	\item \textbf{Work}, in thermodynamics, is any other transfer of energy into or out of a system.
	\item Work and heat refer to energy \textit{in transit}.
	\item The total energy in a system is determined, but not the work nor the heat, it's meaningless.
	\item We ask about how much heat \textit{entered} a system and how much work
	      \textit{was done on} a system.
	\item $\Delta{U} = Q + W$
	      is just a statement of the law of conservation of energy, but it's still called
	      \textbf{first law of thermodynamics}.
	\item Processes of heat transfer: Conduction, Convection, and Radiation.
\end{itemize}

\section{Commpression Work}%
\label{sec:Compression Work}

\begin{itemize}
	\item From classical mmechanics work is $W = \vec{F} \cdot d\vec{r}$
	\item  Consider compressing gas with a piston of area $A$ a distance $\Delta{x}$,
	      the change in volume is $\Delta{V} = -A \Delta{x}$
	\item Volume change should be quasistatic, meaning very slow so that the pressure defined is uniform.
	      then $W = P A \Delta{V}$, but $\Delta{x} = - \Delta{V}$; minus since the volume decreases.
	\item $W =- P A \Delta{V}$  -  quasistatic.
	\item If $P$ is not constant,
	      \begin{align}
		      \label{eq:gas work}
		      W = - \int_{V_i}^{V_f} P(V) \, dV
	      \end{align}

	\item \textbf{isothermal compression} is slow that the temperature doesn't raise.
	\item \textbf{adiabatic compression} is so fast that no heat escapes from the gas.
	\item Isothermal process
	      \begin{itemize}
		      \item  the change will be along an \textbf{ isotherm } line,
		            with $P = NkT / V$.
		      \item The work done is
		            \begin{align}
			            W  = - \int_{V_i}^{V_f} P(V) \, dV = NkT \, \ln{ \frac{V_{i}}{V_{f}} }
		            \end{align}
		      \item The heat enters the system, from the first law, is

		            \begin{align}
			            Q = \Delta{U} - W = \underbrace{ \Delta{( \frac{1}{2}NfkT )} }_{0} - W = NkT \ln{\frac{V_{f}}{V_{i}}}
		            \end{align}
	      \end{itemize}

	\item adiabatic process
	      \begin{itemize}
		      \item In the PV plot the change is from one isotherm to another.
		      \item There should be no trasfer of heat so
		            \begin{align}
			            \Delta{U} = Q+W = W
		            \end{align}
		      \item If it's \textit{ideal} gas, $U$ is proportional to $T$, so the
		            temperature increases.
		      \item By the equipartition theorem $U = \frac{f}{2} NkT$, so $dU = \frac{f}{2} Nk\, dT$,
		            then from \eqref{eq:gas work}
		            \begin{align}
			            \frac{f}{2} Nk \, dT = -P\, dV
		            \end{align}
		            Using the ideal gas law for $P$ and integrate
		            \begin{align}
			            \frac{f}{2} \ln{\frac{T_{f}}{T_{i}}} = - {\frac{V_{f}}{V_{i}}} \quad \text{or} \quad
			            V_{f} T_{f}^{f/2} = V_{i} T_{i}^{f/2} = \text{const.}
		            \end{align}
		      \item Using the ideal gas law to eliminate $T$, $V^{\gamma}P = \text{const.}$,
		            $\gamma$ is the \textbf{adiabatic exponent}.
	      \end{itemize}
\end{itemize}


\section{Heat Capacities}%
\label{sec:Heat Capacities}

\begin{itemize}
	\item \textbf{Heat capacity} of an object is the amount of heat needed to raise its temperature,
	      per degree change
	      \begin{align}
		      \label{eq:heat capacity}
		      C = \frac{Q}{\Delta{T}}
	      \end{align}
	\item The more matter you have the larger the heat capacity, by factoring out the mass $m$
	      we get \textbf{specific heat}
	      \begin{align}
		      \label{eq:specific heat}
		      c \equiv \frac{C}{m}
	      \end{align}
	\item Note \eqref{eq:specific heat} is ambiguous, plug in the first law
	      \begin{align}
		      \label{eq:heat cap ambig}
		      C = \frac{Q}{\Delta{T}} = \frac{\Delta{U} - W}{\Delta{T}}
	      \end{align}
	      Even if the energy of an object is a well-defined function of its temperature alone,
	      the work $W$ done on the object is not; it depends on the process path on PV plot.
	\item From \eqref{eq:heat cap ambig} The \textbf{heat capacity at constant volume}, denoted $C_{V}$
	      \begin{align}
		      \label{eq:heat cap v}
		      C_{V} = {\left( \frac{\Delta{U}}{\Delta{T}} \right)}_{V} =
		      \left( \frac{\partial U}{\partial T} \right)_{V}
	      \end{align}
	\item From \eqref{eq:heat cap ambig} and \eqref{eq:gas work}
	      the \textbf{heat capacity at constant pressure}, denoted $C_{P}$
	      \begin{align}
		      \label{eq:heat cap p}
		      C_{P} = \left( \frac{\Delta{U} - (- P \Delta{V})}{\Delta{T}}  \right)_{P}
		      = \left ( \frac{\partial U}{\partial T}  \right)_{P}
		      + P \left ( \frac{\partial V}{\partial T}  \right)_{P}
	      \end{align}
	      for solid last term is almost negligible.
	\item At a \textbf{phase transformation}, you add heat in a system without increasing its temperature;
	      such as melting of boiling water. Then $C = \frac{Q}{\Delta{T}} = \infty$
	\item The amount of heat needed to do this phase transformation is called \textbf{latent heat} $L$,
	      and the \textbf{specific latent heat} is
	      \begin{align}
		      \label{eq:specific latent}
		      l \equiv \frac{L}{m}= \frac{Q}{m}
	      \end{align}
	      It's ambiguous, but we assume the pressure is constant, and no other work done.
	\item Adding $PV$ onto the energy gives the \textbf{enthalpy}
	      \begin{align}
		      \label{eq:enthalpy}
		      H = U + PV
	      \end{align}
	      it's the \textit{total} energy you would need to create the system out of nothing.



\end{itemize}


\chapter{The Second Law}

\section{Two-State system}%
\label{sec:two state}
\begin{itemize}

	\item Irreversible processes are not \textit{inevitable}, they are just overwhelmingly \textit{probable}.
	\item A \textbf{microstate} is an outcome.
	\item A \textbf{macrostate} is saying number of particle for each state, e.g., two heads.
	\item \textbf{multiplcity} of a macrostate is the number of microstates for a given macrostate.
	\item Total multiplicity of all macrostates is the total number of microstates. Then the probability
	      of a macrostate is
	      \begin{align}
		      p = \frac{\Omega(n)}{\Omega(\text{all})}
	      \end{align}
	\item Number of different ways of choosing $n$ items out of $N$, or the \textit{combination} of $n$
	      chosen from $N$.
	      \begin{align}
		      \label{eq:multiplicity formula}
		      \Omega(N, n) = \frac{N!}{n! \, (N-n)!} =
		      \begin{pmatrix}
			      N \\ n
		      \end{pmatrix}
	      \end{align}
\end{itemize}



\section{The Einstein Model of a Solid}%
\label{sec:einstein model}

Consider a collection of microscopic system that can each store any number of energy `units'
Equal-size energy units as in quantum harmonic oscillator.
\begin{itemize}
	\item In three-dimensional solid, each atom can oscillate in three independent directions, for $N$
	      oscillators there are $N/3$ atoms.
	\item The multiplicity of an Einstein solid with $N$ oscillators and $q$
	      energy units is
	      \begin{align}
		      \label{eq:einstein omega}
		      \Omega(N, q) = \begin{pmatrix}
			                     q + N -1 \\ q
		                     \end{pmatrix} =
		      \frac{(q+N-1)!}{q! \, (N-1)!}
	      \end{align}
\end{itemize}


\section{Interacting Systems}%
\label{sec:Interacting Systems}

To understand the heat flow and irreversible processes, consider two Einstein solids that can
share energy back and forth

\begin{itemize}
	\item Assume the two solids are \textbf{weakly coupled}; the exchange of energy between them
	      is slower then the exchange of energy among atom within each solid.
	\item Then the individual energies of solids, $U_A$ and $U_B$, will change slowly;
	      over short time they are fixed.
	\item on longer time scales the values of $U_A$ and $U_B$ will change, so we consider
	      $U_\text{total} = U_A + U_A$
	\item The total multiplicity for independent system is just the product of $\Omega_A$ and $Omega_B$.
	\item \textbf{Fundamental assumption of statistical mechanics} In an isolated system
	      in thermal equilibrium, all accessible microstates are equally probable.
	\item Invoking this principle on the two Einstein solids, we conclude that, while all the
	      \textit{micro}states are equally probable, some \textit{macro}states are more probable than others.
	\item The \textit{heat}, then, is a probabilistic phenomenon; if system started \textit{initially}
	      with all the energy in solid $B$ and wait for a while, it's more certain to find that the
	      energy has flowed from $B$ to $A$.
	\item The tendency of increasing multiplicity is the \textbf{second law of thermodynamics}.
\end{itemize}


\section{Large Systems}%
\label{sec:large systems}

\begin{itemize}
	\item There three categories of numbers here:
	      \begin{itemize}
		      \item	\textbf{small numbers}, 12, 43
		      \item	\textbf{large numbers}, in order of Avogadro's number \num{d23}
		      \item	\textbf{very large numbers}, exponentiating of large numbers.
	      \end{itemize}
	\item Adding a small number to a large number doesn't change.
	      Multiplying a large number by a very large number.
	\item Using Stirling's approximation for factorial of a large number
	      \begin{align}
		      \label{eq:Stirling}
		      N! \approx N^{N} e^{-N} \sqrt{2 \pi N} ; \qquad  N \gg 1
	      \end{align}
	\item The approximated multiplicity is
	      \begin{align}
		      \label{eq:appox omega}
		      \Omega (N, q) \approx \left( \frac{eq}{N} \right)^{N}; \qquad q \gg N
	      \end{align}
	      within the high-temperature limit.
	\item Multiplicity of two Einstein interacting solids with $N$ oscillators and total $q$ energy units is
	      \begin{align}
		      \label{eq:total multipliciy}
		      \Omega = \left( \frac{eq_{A}}{N} \right)^{N}  \left( \frac{eq_B}{N} \right)^{N}
		      = \left( \frac{e}{N} \right)^{2N} (q_A q_B)^N
	      \end{align}
	\item Taking $\Omega = \Omega(q_A)$, it has a peak at $q_A= q/2$ with a very large number value of
	      \begin{align}
		      \label{eq:Omega max}
		      \Omega_{\text{max}} = \left( \frac{e}{N} \right)^{2N} \left(\frac{q}{2} \right)^{2N}
	      \end{align}

	\item Around this maximum, $\Omega$ is a \textbf{Gaussian} as
	      \begin{align}
		      \Omega = \Omega_{\text{max}} \, e^{-N(2x/q)^2}
	      \end{align}
	\item  $\Omega = \Omega_{\text{max}}/e$ when
	      \begin{align*}
		      x = \frac{q}{2 \sqrt{N}}
	      \end{align*}
	      the width is $q/\sqrt{N}$
\end{itemize}







\end{document}
