\documentclass{book}

\usepackage[a4paper, lmargin=20mm, rmargin=20mm]{geometry}
\usepackage{amsmath}
\usepackage{amsthm}
\usepackage{amssymb}
\usepackage{graphicx}
\usepackage{siunitx}

\title{PHYS430 - Thermal Physics}
\author{Alfaifi, Ammar}
\date{}

\begin{document}

\maketitle

\chapter{Energy in Thermal Physics}
\section{Thermal Eequilibrium}%
\label{sec:thermal-equilibrium}

\begin{itemize}
  \item After two objects have been in contact long enough, we say that they are in \textbf{thermal equilibrium}.
  \item The time required for a system to come to thermal equilibrium is called the \textbf{relaxation time}.
  \item \textbf{Temperature} is a measure of the tendency of an object to spontaneously give up energy to its
        surroundings.
  \item The flow of energy is from the object with a higher temperature to the lower on.
  \item For low-density gas at constant pressure, the volume should go to \textit{zero} at
        approximately $-273^{\circ}$C. which defines the \textbf{absolute zero}, in the
        \textbf{absolute temperature scale}, in K (kelvin).
\end{itemize}

\section{The Ideal Gas}%
\label{sec:The Ideal Gas}
\begin{align}
  PV = nRT; \qquad R = \qty{8.31}{J / mol . K}
\end{align}

\begin{itemize}
  \item A \textbf{mole} of molecules is Avogadro's number of them, \num{6.02d23}.
  \item Number of molecules is $N=n n_{A}$
  \item Ideal gas law becomes $PV = NkT$, where $k$ is Boltzmann's constant.

\end{itemize}

\end{document}
