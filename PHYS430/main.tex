\documentclass{book}

\usepackage[a4paper, lmargin=20mm, rmargin=20mm]{geometry}
\usepackage{amsmath}
\usepackage{amsthm}
\usepackage{amssymb}
\usepackage{graphicx}
\usepackage{siunitx}

\title{PHYS430 - Thermal Physics}
\author{Alfaifi, Ammar}
\date{}

\begin{document}

\maketitle

\chapter{Energy in Thermal Physics}


\section{Thermal Equilibrium}%
\label{sec:thermal-equilibrium}

\begin{itemize}
  \item After two objects have been in contact long enough, we say that they are in \textbf{thermal equilibrium}.
  \item The time required for a system to come to thermal equilibrium is called the \textbf{relaxation time}.
  \item \textbf{Temperature} is a measure of the tendency of an object to spontaneously give up energy to its
        surroundings.
  \item The flow of energy is from the object with a higher temperature to the lower on.
  \item For low-density gas at constant pressure, the volume should go to \textit{zero} at
        approximately $-273^{\circ}$C. which defines the \textbf{absolute zero}, in the
        \textbf{absolute temperature scale}, in K (kelvin).
\end{itemize}

\section{The Ideal Gas}%
\label{sec:The Ideal Gas}
\begin{align}
  PV = nRT; \qquad R = \qty{8.31}{J / mol . K}
\end{align}

\begin{itemize}
  \item A \textbf{mole} of molecules is Avogadro's number of them, \num{6.02d23}.
  \item Number of molecules is $N=n \times N_{A}$
  \item Ideal gas law becomes $PV = NkT$, where $k$ is Boltzmann's constant.
  \item The average transnational kinetic energy is $\bar{K}_{\text{trans}}= \frac{3}{2}kT$,
        where $kT = \frac{1}{40} \si{\electronvolt}$
\end{itemize}

\section{Equipartition of Energy}%
\label{sec:equi of energy}

\paragraph{Equipartition theorem} At a temperature $T$, the average energy of any
quadratic degree of freedom is $\frac{1}{2}kT$.

For a system of $N$ molecules, each with $f$ degree of freedom, and there are no other
(non-quadratic) temperature-dependent forms of energy, then its \textbf{total thermal energy} is

\begin{align}
U = Nf \frac{1}{2}kT
\end{align}
Note, This is the \textit{average} total thermal energy,
but for large $N$, fluctuations become negligible.


\section{Heat and Work}%
\label{sec:heat and work}

\begin{itemize}
  \item Total amount of energy in the universe never changes, \textbf{Conservation of energy}
  \item \textbf{Heat} any spontaneous flow of energy form on e object to another, caused by
        difference in temperature.
  \item \textbf{Work}, in thermodynamics, is any other transfer of energy into or out of a system.
  \item Work and heat refer to energy \textit{in transit}.
  \item The total energy in a system is determined, but not the work nor the heat, it's meaningless.
  \item We ask about how much heat \textit{entered} a system and how much work
        \textit{was done on} a system.
  \item $\Delta{U} = Q + W$
        is just a statement of the law of conservation of energy, but it's still called
        \textbf{first law of thermodynamics}.
  \item Processes of heat transfer: Conduction, Convection, and Radiation.
\end{itemize}

 \section{Commpression Work}%
 \label{sec:Compression Work}

\begin{itemize}
 \item From classical mmechanics work is $W = \vec{F} \cdot d\vec{r}$
 \item  Consider compressing gas with a piston of area $A$ a distance $\Delta{x}$,
   the change in volume is $\Delta{V} = -A \Delta{x}$
 \item Volume change should be quasistatic, meaning very slow so that the pressure defined is uniform.
     then $W = P A \Delta{V}$, but $\Delta{x} = - \Delta{V}$; minus since the volume decreases.
 \item $W =- P A \Delta{V}$  -  quasistatic.
  \item If $P$ is not constant,
\begin{align}
\label{eq:gas work}
        W = - \int_{V_i}^{V_f} P(V) \, dV
\end{align}

  \item \textbf{isothermal compression} is slow that the temperature doesn't raise.
  \item \textbf{adiabatic compression} is so fast that no heat escapes from the gas.
  \item Isothermal process
        \begin{itemize}
          \item  the change will be along an \textbf{ isotherm } line,
                with $P = NkT / V$.
          \item The work done is
        \begin{align}
          W  = - \int_{V_i}^{V_f} P(V) \, dV = NkT \, \ln{ \frac{V_{i}}{V_{f}} }
        \end{align}
          \item The heat enters the system, from the first law, is

                \begin{align}
                  Q = \Delta{U} - W = \underbrace{ \Delta{( \frac{1}{2}NfkT )} }_{0} - W = NkT \ln{\frac{V_{f}}{V_{i}}}
                \end{align}
    \end{itemize}

  \item adiabatic process
        \begin{itemize}
          \item In the PV plot the change is from one isotherm to another.
          \item There should be no trasfer of heat so
                \begin{align}
                  \Delta{U} = Q+W = W
                \end{align}
          \item If it's \textit{ideal} gas, $U$ is proportional to $T$, so the
                temperature increases.
          \item By the equipartition theorem $U = \frac{f}{2} NkT$, so $dU = \frac{f}{2} Nk\, dT$,
                then from \eqref{eq:gas work}
                \begin{align}
                  \frac{f}{2} Nk \, dT = -P\, dV
                \end{align}
                Using the ideal gas law for $P$ and integrate
                \begin{align}
                  \frac{f}{2} \ln{\frac{T_{f}}{T_{i}}} = - {\frac{V_{f}}{V_{i}}} \quad \text{or} \quad
                  V_{f} T_{f}^{f/2} = V_{i} T_{i}^{f/2} = \text{const.}
                \end{align}
          \item Using the ideal gas law to eliminate $T$, $V^{\gamma}P = \text{const.}$,
                $\gamma$ is the \textbf{adiabatic exponent}.
        \end{itemize}
\end{itemize}


\section{Heat Capacities}%
\label{sec:Heat Capacities}

\begin{itemize}
  \item \textbf{Heat capacity} of an object is the amount of heat needed to raise its temperature,
        per degree change
        \begin{align}
          \label{eq:heat capacity}
          C = \frac{Q}{\Delta{T}}
        \end{align}
  \item The more matter you have the larger the heat capacity, by factoring out the mass $m$
        we get \textbf{specific heat}
        \begin{align}
          \label{eq:specific heat}
          c \equiv \frac{C}{m}
        \end{align}
  \item Note \eqref{eq:specific heat} is ambiguous, plug in the first law
        \begin{align}
          \label{eq:heat cap ambig}
          C \frac{Q}{\Delta{T}} = \frac{\Delta{U} - W}{\Delta{T}}
        \end{align}
        Even if the energy of an object is a well-defined function of its temperature alone,
        the work $W$ done on the object is not; it depends on the process path on PV plot.
  \item From \eqref{eq:heat cap ambig} The \textbf{heat capacity at constant volume}, denoted $C_{V}$
        \begin{align}
          \label{eq:heat cap v}
          C_{V} = \left( \frac{\Delta{U}}{\Delta{T}} \right) _{V} =
          \left( \frac{\partial U}{\partial T} \right)_{V}
        \end{align}
  \item From \eqref{eq:heat cap ambig} and \eqref{eq:gas work}
        the \textbf{heat capacity at constant pressure}, denoted $C_{P}$
        \begin{align}
          \label{eq:heat cap p}
          C_{P} = \left( \frac{\Delta{U} - (- P \Delta{V})}{\Delta{T}}  \right)_{P}
          = \left ( \frac{\partial U}{\partial T}  \right)_{P}
            + P \left ( \frac{\partial V}{\partial T}  \right)_{P}
        \end{align}
        for solid last term is almost negligible.
  \item At a \textbf{phase transformation}, you add heat in a system without increasing its temperature;
        such as melting of boiling water. Then $C = \frac{Q}{\Delta{T}} = \infty$
  \item The amount of heat needed to do this pahse transformation is called \textbf{latent heat} $L$,
        and the \textbf{specific latent heat} is
        \begin{align}
          \label{eq:specific latent}
          l \equiv \frac{L}{m}= \frac{Q}{m}
        \end{align}
        It's ambiguous, but we assume the pressure is constant, and no other work done.
  \item Adding $PV$ onto the energy gives the \textbf{enthalpy}
        \begin{align}
          \label{eq:enthalpy}
          H = U + PV
        \end{align}
        it's the \textit{total} energy you would need to create the system out of nothing.



\end{itemize}


\chapter{The Second Law}

\section{Two-State system}%
\label{sec:two state}



\end{document}
